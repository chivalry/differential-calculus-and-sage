% This is part of the book TeX for the Impatient.
% Copyright (C) 2003 Paul W. Abrahams, Kathryn A. Hargreaves, Karl Berry.
% Copyright (C) 2004 Marc Chaudemanche pour la traduction fran�aise.
% See file fdl.tex for copying conditions.

\input fmacros
\chapter {Commandes pour \linebreak les modes horizontaux \linebreak et verticaux}

\chapterdef{hvmodes}

Cette section couvre les commandes qui ont des formes correspondantes ou 
identiques dans les modes horizontaux et verticaux. 
Ces commandes fournissent des bo\^\i tes, des espaces, des filets, des r\'eglures 
et des alignements. 
Pour une explication des conventions utilis\'ees dans cette section, 
voir \headcit{Descriptions des commandes}{cmddesc}.

\begindescriptions

%==========================================================================
\section {Produire des espaces}

%==========================================================================

\subsection {Espaces horizontaux de largeur fixe}

\begindesc
\bix^^{espace//production d'}
\bix^^{espace horizontal}
\cts thinspace {}
\explain
Cette commande produit un \minref{cr\'enage} positif
dont la largeur \'equivaut \`a un sixi\`eme d'em
(\xref{dimdefs})
c'est-\`a-dire qu'il demande \`a \TeX\ de d\'eplacer sa position \`a droite de ce montant.
C'est pratique
quand vous avez par exemple des quotes imbriqu\'ees
et que vous voulez les s\'eparer.
\TeX\ ne coupe pas de ligne sur un
|\thinspace|.\minrefs{line break}

\example
``\thinspace`A quote.'\thinspace''\par
24,\thinspace 29--31,\thinspace 45,\thinspace 102
|
\produces
``\thinspace`A quote.'\thinspace''\par
24,\thinspace 29--31,\thinspace 45,\thinspace 102
\endexample
\enddesc

\begindesc
\cts negthinspace {}
\explain
Cette commande produit un \minref{cr\'enage} n\'egatif
dont la largeur \'equivaut \`a un sixi\`eme d'em
(\xref{dimdefs})
c'est-\`a-dire qu'il demande \`a \TeX\ de d\'eplacer sa position \`a gauche de ce montant.
C'est pratique pour rapprocher deux caract\`eres qui sont un peu trop
\'eloign\'es l'un de l'autre.
\TeX\ ne coupe pas de ligne sur un |\negthinspace|.\minrefs{line break}
\example
The horror, the horror\negthinspace, the horror of it all!!
|
\produces
The horror, the horror\negthinspace, the horror of it all!
\endexample
\enddesc

\begindesc
\cts enspace {}
\explain
Cette commande produit un \minref{cr\'enage} dont la largeur \'equivaut \`a un ^{en}
(la moiti\'e d'un em, voir \xrefpg{dimdefs}).
\TeX\ ne coupe pas de ligne sur un |\enspace| \`a moins qu'il soit suivi d'un ressort.
Dans une liste \`a puce, les puces sont g\'en\'eralement s\'epar\'ees du texte qui les suit 
par un |\enspace|.
\example
Lemma 1.\enspace There exists a white rabbit.
|
\produces
Lemma 1.\enspace There exists a white rabbit.
\endexample
% the enspace here follows the \proclaim convention (Knuth p. 355).
\enddesc

\begindesc
\easy\cts enskip {}
\cts quad {}
\cts qquad {}
\explain
^^{en}
Chacune de ces commandes 
produit un bout de \minref{ressort} horizontal
qui ne peut ni s'\'etirer ni se r\'etr\'ecir.
\TeX\ peut couper une ligne \minrefs{line break} sur un tel ressort.
Les largeurs de ces ressort/dimension
(qui sont relatifs \`a la police courante)
sont les suivantes pour |cmr10|, la police par d\'efaut de \plainTeX~:
\medskip
\def\distance#1{\hbox{\strut $\rightarrow$\vrule\hskip #1\vrule$\leftarrow$}}
{\def\l#1{\hfil$#1$\thinspace em}
\tabskip 3pc \halign{#&#&\hfil#\hfil\cr
{\it Commande}&{\it Espace}&{\it Illustration}\cr
\noalign{\vskip 4pt}
|\enskip|&\l{\frac 1/2}&\distance{.5em}\cr
|\quad|&\l1&\distance{1em}\cr
|\qquad|&\l2&\distance{2em}\cr
}}
\example
en\enskip skip; quad\quad skip; qquad\qquad skip
|
\produces
en\enskip skip; quad\quad skip; qquad\qquad skip
\endexample
\eix^^{espace horizontal}
\enddesc

%==========================================================================
\subsection {Espaces verticaux de largeur fixe}

\begindesc
\bix^^{espace vertical}
\easy\cts smallskip {}
\cts medskip {}
\cts bigskip {}
\explain
Ces commandes produisent des montants d'espacement vertical 
successivement plus larges~:
\display{{\def\bar{\kern 2pt\hrule width 5pc} 
\def\lbl#1{\hbox to 5pc{\hfil #1skip\hfil}}
\leavevmode
\vtop{\lbl{small}\bar\smallskip\bar}\quad
\vtop{\lbl{med}\bar\medskip\bar}\quad\vtop{\lbl{big}\bar\bigskip\bar}}}
\noindent
|\smallskip| saute de $3$ points et peut s'\'etirer ou
s'\'elargir de $1$ point.  |\med!-skip| est \'equivalent \`a deux |\smallskip|s
et |\bigskip| est \'equivalent \`a deux |\medskip|s.

Ces commandes terminent un paragraphe car elles sont fondamentalement verticales.
Les sauts qu'elles produisent
sont ajout\'es au saut inter-para\-graphe normal.
\example
Hop \smallskip skip \medskip and \bigskip jump.
|
\produces
Hop \smallskip skip \medskip and \bigskip jump.
\endexample
\enddesc

\begindesc
\cts smallskipamount {\param{ressort}}
\cts medskipamount {\param{ressort}}
\cts bigskipamount {\param{ressort}}
\explain
Ces param\`etres sp\'ecifient les montants de ressort produit par
les commandes |\small!-skip|, |\med!-skip| et |\big!-skip|.
En changeant ces param\`e\-tres vous changez l'effet des commandes.
Les valeurs par d\'efaut (pour \plainTeX)
correspondent \`a un quart d'une ligne d'espace, \`a la moiti\'e d'une ligne d'espace,
et \`a une ligne d'espace enti\`ere.
Nous vous recommandons de maintenir ce ratio en changeant
ces valeurs \`a chaque fois que
vous changez |\baselineskip| (\xref\baselineskip).
^^|\baselineskip//et {\tt\\smallskipamount}, etc.|
\eix^^{espace vertical}

\enddesc

%==========================================================================
\subsection {Espace de taille variable}

\begindesc
\easy\cts hskip {\<dimension$_1$> {\bt plus} \<dimension$_2$> {\bt minus} \<dimension$_3$>}
\cts vskip {\<dimension$_1$> {\bt plus} \<dimension$_2$> {\bt minus} \<dimension$_3$>}
^^{saut vertical}^^{ressort vertical}
^^{saut horizontal}^^{ressort horizontal}
\bix^^{espace horizontal}
\bix^^{espace vertical}
\explain
Ces commandes produisent des ressorts respectivement horizontaux et verticaux.
Dans le cas le plus simple et le plus courant, quand il n'y a seulement que \<dimension$_1$>,
|\hskip| saute vers la droite de \<dimension$_1$>
et |\vskip| saute vers le bas de la page de \<dimension$_1$>.
Plus g\'en\'erale\-ment, ces commandes
produisent des \minref{ressort}s dont la taille naturelle est
\<dimension$_1$>, l'\'etirement \<dimension$_2$> et le r\'etr\'ecissement \<dimension$_3$>.
Le |plus| \<dimension$_2$>, le |minus |\<dimension$_3$>,
ou les deux peuvent \^etre omis.
Si les deux sont pr\'esent, le |plus| doit \^etre avant le |minus|.
Une valeur omise
est consid\'er\'ee \`a z\'ero. Toutes les \<dimension>s peuvent \^etre n\'egatives.

Vous pouvez utiliser |\hskip| en mode math\'ematique, mais vous ne pouvez utiliser d'unit\'es |mu|
\seeconcept{unit\'e math\'ematique}
pour aucune des dimensions.  Si vous voulez des unit\'es |mu|, utilisez |\mskip|
(\xref\mskip) \`a la place.

\example
\hbox to 2in{one\hskip 0pt plus .5in two}
|
\produces
\hbox to 2in{one\hskip 0pt plus 2in two}

\doruler{\8\8}2{in}
\nextexample
\hbox to 2in{Help me!! I can't fit
{\hskip 0pt minus 2in} inside this box!!}
|
\produces
\hbox to 2in{Help me! I can't fit
{\hskip 0pt minus 2in} inside this box!}

\doruler{\8\8}2{in}
\nextexample
\vbox to 4pc{\offinterlineskip% Just show effects of \vskip.
   \hbox{one}\vskip 0pc plus 1pc \hbox{two}
      \vskip .5pc \hbox{three}}
|
\produces
\smallskip
\vbox to 4pc{\offinterlineskip% Just show effects of \vskip.
   \hbox{one}\vskip 0pc plus 1pc \hbox{two}
      \vskip .5pc \hbox{three}}
\endexample
\enddesc

\begindesc
\cts hglue {\<ressort>}
\cts vglue {\<ressort>}
^^{ressort vertical}
^^{ressort horizontal}
\explain
La commande |\hglue| produit un \minref{ressort} horizontal qui ne dispara\^\i t pas sur
une coupure de ligne. La commande |\vglue|
produit un \minref{ressort} vertical qui ne dispara\^\i t pas sur une coupure de page. 
Pour le reste ces commandes sont comme |\hskip| et |\vskip|.
Vous pouvez utiliser |\vglue| pour produire de l'espace blanc en haut d'une page,
par exemple, au dessus d'un titre sur la premi\`ere page d'un document, mais
^|\topglue| (suivant) est normalement plus appropri\'e pour cet usage.
\enddesc

\begindesc
\cts topglue {\<ressort>}
\explain
\margin{Command added; recent addition to \TeX}
Cette commande\footnote{|\topglue| a \'et\'e ajout\'e \`a \TeX\ dans la version 3.0,
apr\`es les autres am\'enagements introduitis par le ^{\newTeX}
(\xref{newtex}).  C'est la premi\`ere d\'ecrite dans la \emph{dix-huiti\`eme\/}
\'edition de \texbook.} fait que l'espace entre le haut de la page et le
haut de la premi\`ere bo\^\i te de la page soit pr\'ecis\'ement \'egal \`a \<glue>.
Le haut de la page est consid\'er\'e \^etre sur la ligne de base d'une
ligne imaginaire de texte juste au-dessus de la ligne du haut de la page.
Plus pr\'ecis\'ement, c'est une distance |\topskip| au dessus de l'origine donn\'ee par
|\hoffset| et |\voffset|.

Cette commande est pratique parce que \TeX\ ajuste le ressort normalement 
produit par |\topskip| d'une mani\`ere complexe.  En utilisant |\top!-glue| vous pouvez
contr\^oler la position de la premi\`ere bo\^\i te de la page sans vous inqui\'eter de 
ces ajustements.

\enddesc

\begindesc
\cts kern {\<dimension>}
\explain
L'effet de cette commande d\'epend du mode dans lequel est \TeX\ quand 
il la rencontre~:
\ulist

\li En mode horizontal, \TeX\ d\'eplace sa position vers la droite (pour un cr\'enage positif) 
ou vers la gauche (pour un cr\'enage n\'egatif).

\li En mode vertical, \TeX\ d\'eplace sa position vers le bas de la page (pour un cr\'enage positif)
ou vers le haut de la page (pour un cr\'enage n\'egatif).

\endulist
\noindent
Ainsi un cr\'enage positif produit un espace vide tandis qu'un cr\'enage n\'egatif
fait que \TeX\ recule sur quelque chose qu'il a d\'ej\`a produit.
Cette notion de cr\'enage ^^{cr\'enages}
est diff\'erente de la notion de cr\'enage dans certains syst\`emes de composition 
typographique informatis\'es---dans \TeX, des cr\'e\-nages positifs  \emph{\'ecartent} 
deux lettres au lieu de les rapprocher.

Un cr\'enage est similaire \`a un
\minref{ressort}, sauf que (a)~un cr\'enage ne peut ni s'\'etirer ni se r\'etr\'ecir,
et (b)~\TeX\ ne coupera une ligne ou une page sur un cr\'enage que si le cr\'enage
est suivi par un ressort et ne fait pas partie d'une formule math\'ematique.
Si \TeX\ trouve un cr\'enage \`a la fin 
d'une ligne ou d'une page, il \'elimine le cr\'enage.
Si vous voulez obtenir l'effet d'un cr\'enage qui ne dispara\^\i t jamais,
utilisez ^|\hglue| ou ^|\vglue|.

Vous pouvez utiliser |\kern| en mode math\'ematique, mais vous ne pouvez pas utiliser d'unit\'es |mu|
\seeconcept{unit\'e math\'ematique}
pour \<dimension>.  Si vous voulez des unit\'es |mu|, utilisez |\mkern|
(\xref\mkern) \`a la place.
^^{coupures de ligne//cr\'enages sur des}
^^{coupures de page//cr\'enages sur des}

\example
\centerline{$\Downarrow$}\kern 3pt % a vertical kern
\centerline{$\Longrightarrow$\kern 6pt % a horizontal kern
   {\bf Heed my warning!!}\kern 6pt % another horizontal kern
   $\Longleftarrow$}
\kern 3pt % another vertical kern
\centerline{$\Uparrow$}
|
\produces
\centerline{$\Downarrow$}\kern 3pt % a vertical kern
\centerline{$\Longrightarrow$\kern 6pt % a horizontal kern
   {\bf Heed my warning!}\kern 6pt % another horizontal kern
   $\Longleftarrow$}
\kern 3pt % another vertical kern
\centerline{$\Uparrow$}
\endexample
\enddesc

\begindesc
\makecolumns 4/2:
\cts hfil {}
\cts hfill {}
\cts vfil {}
\cts vfill {}
\explain
\bix^^{ressort//\'etirable infiniment}
Ces commandes produisent des ressorts horizontaux et verticaux \'etirables 
infiniment qui
\'ecrasent tout \'etirement fini qui peut \^etre pr\'esent.
|\hfil| et |\hfill| produisent des ressorts horizontaux, tandis que
|\vfil| et |\vfill| produisent des ressorts verticaux.

|\hfill| est infiniment plus grand que |\hfil|.
Si |\hfill| et |\hfil| apparaissent dans la m\^eme \minref{bo\^\i te},
le |\hfill| consumera toute la place suppl\'ementaire possible
et le |\hfil| sera effectivement ignor\'e.
|\hfill| peut \`a son tour est \'ecras\'e par |\hskip 0pt plus 1filll|.
Les ressorts produits par
|\hfil| et |\hfill| ne r\'etr\'ecissent jamais.

Le comportement de |\vfil| et |\vfill| est analogue.

\example
\hbox to 2in{Left\hfil Middle \hfil Right}
|
\produces
\hbox to 2in{Left\hfil Middle \hfil Right}


\doruler{\8\8}2{in}
\nextexample
\hbox to 2in{Left\hfil Middle \hfill Right}
|
\produces
\hbox to 2in{Left\hfil Middle \hfill Right}

\doruler{\8\8}2{in}
\nextexample
\leftline{%
\vbox to 4pc{%
   \hbox{Top}\vfil\hbox{Middle}\vfil \hbox{Bottom}}\quad
\vbox to 4pc{%
   \hbox{Top}\vfil\hbox{Middle}\vfill\hbox{Bottom}}}
|
\produces
\smallskip
\leftline{%
\vbox to 4pc{%
   \hbox{Top}\vfil\hbox{Middle}\vfil \hbox{Bottom}}\quad
\vbox to 4pc{%
   \hbox{Top}\vfil\hbox{Middle}\vfill\hbox{Bottom}}}
\endexample

\eix^^{ressort//\'etirable infiniment}
\enddesc

\begindesc
\cts hss {}
\cts vss {}
\explain
Ces commandes produisent des ressorts horizontaux et 
verticaux qui sont infiniment \'etirables et 
r\'etr\'ecissables. Le ressort peut se r\'etr\'ecir 
sur une distance n\'egative, produisant l'effet de reculer sur une ligne 
(pour |\hss|) ou remonter sur une page (pour |\vss|).

\example
\line{text\hfil\hbox to 0pt{margin\hss}}
% `margin\hss' shrinks to the zero width of the hbox.
|
\produces
\line{text\hfil\hbox to 0pt{margin\hss}}
\nextexample
\vbox to 1pc{\hrule width 6pc % Top of box.
   \hbox{1} \vskip 1pc\hbox to 2pc{\hfil 2}
   % The \vss absorbs the extra distance produced by \vskip.
   \vss \hbox to 3pc{\hfil 3}
   \hrule width 6pc}% Bottom of box.
|
\produces
\medskip
\vbox to 1pc{\hrule width 6pc % top of box
   \hbox{1} \vskip 1pc\hbox to 2pc{\hfil 2}
   \vss \hbox to 3pc{\hfil 3}
   \hrule width 6pc}% bottom of box
\vskip 2.5pc
\endexample
\enddesc

\begindesc
\cts hfilneg {}
\cts vfilneg {}
\explain
^^{ressort//n\'egatif}
Ces commandes effacent l'effet d'un |\hfil| ou d'un |\vfil| 
pr\'ec\'edent. Tandis que |\hfil| et |\vfil| produisent un 
\minref{ressort} positif infiniment \'etirable, |\hfilneg| et |\vfilneg| 
produisent un ressort n\'egatif infiniment \'etirable. 
(ainsi, $n$ |\hfilneg| effacent $n$ ^|\hfil| et pareillement 
pour le |\vfilneg|.) 
L'utilisation principale de |\hfilneg| et de |\vfilneg| est de contrecarrer 
l'effet d'un |\hfil| ou d'un |\vfil| ins\'er\'e par une \minref{macro}.

|\hfilneg| et |\vfilneg| ont 
la propri\'et\'e curieuse, s'ils sont les seuls ressorts infiniment \'etirables 
d'une bo\^\i te, de produire exactement le m\^eme effet que |\hfil| 
et |\vfil|.

\example
\leftline{\hfil on the right\hfilneg}
% Cancel the \hfil that \leftline produces to the right
% of its argument.
|
\produces
\leftline{\hfil on the right \hfilneg}
% Cancel the \hfil that \leftline produces to the right
% of its argument.
%
\nextexample
\def\a{\hbox to 1pc{\hfil 2}\vfil}
\vbox to 4pc{\hbox{1} \vfil \a
   \vfilneg \hbox to 2pc{\hfil 3}}
|
\produces
\smallskip
\def\a{\hbox to 1pc{\hfil 2}\vfil}
\vbox to 4pc{\hbox{1} \vfil \a
   \vfilneg \hbox to 2pc{\hfil 3}}
\endexample\enddesc

{\emergencystretch=2em
\see |\hbadness| et |\vbadness| (\xref \vbadness),
|\hfuzz| et |\vfuzz| (\xref \vfuzz),
\conceptcit{filets}.\par}
\eix^^{espace//production d'}
\eix^^{espace horizontal}
\eix^^{espace vertical}

%==========================================================================
\section {Manipuler des bo\^\i tes}

%==========================================================================
\subsection {Construire des hbox et des vbox}

\begindesc
\bix^^{commandes de \boites}
^^{hbox//construction avec \b\tt\\hbox\e}
%
\cts hbox {\rqbraces{\<mat\'eriel en mode horizontal>}}
\aux\cts hbox {{\bt to} \<dimension> \rqbraces{\<mat\'eriel en mode horizontal>}}
\aux\cts hbox {{\bt spread} \<dimension> \rqbraces{\<mat\'eriel en mode horizontal>}}
\explain
Ces commandes produisent une hbox
(\minref{bo\^\i te} horizontale) contenant du \<mat\'eriel en mode horizontal>.
Les accolades autour du \<mat\'eriel en mode horizontal> d\'efinissent un groupe.
\TeX\ ne coupe pas le \<mat\'eriel en mode horizontal> en plusieurs lignes,
puisqu'il est en mode horizontal restreint quand il assemble la bo\^\i te.
\TeX\ ne change pas la taille de la bo\^\i te une fois qu'elle est produite.

|\hbox| est souvent utile quand vous voulez conserver du texte sur une seule ligne.
Si votre utilisation de |\hbox| emp\^eche \TeX\ de couper des lignes de mani\`ere acceptable, 
\TeX\ se plaindra d'un ``overfull hbox''.

La largeur du hbox d\'epend des arguments de |\hbox|~:
\ulist\compact
\li Si vous ne sp\'ecifiez que du \<mat\'eriel en mode horizontal>,
le hbox aura sa \minref{largeur} naturelle.
\li Si vous sp\'ecifiez |to| \<dimension>, la largeur du hbox sera \<dimension>.
\li Si vous sp\'ecifiez |spread| \<dimension>, la largeur du hbox sera
sa largeur naturelle plus \<dimension>, c'est-\`a-dire, le hbox s'allongera de
\<dimension>.
\endulist

La commande ^|\hfil| (\xref\hfil) est utile pour remplir un
hbox d'espace vide quand le mat\'eriel dans la bo\^\i te n'est pas aussi large que la
largeur de la bo\^\i te.
\example
\hbox{ugly suburban sprawl}
\hbox to 2in{ugly \hfil suburban \hfil sprawl}
\hbox spread 1in {ugly \hfil suburban \hfil sprawl}
% Without \hfil in the two preceding lines,
% you'd get `underfull hbox'es.
|
\produces
\hbox{ugly suburban sprawl}
\hbox to 2in{ugly \hfil suburban \hfil sprawl}
\hbox spread 1in {ugly \hfil suburban \hfil sprawl}
% Without \hfil in the two preceding lines,
% you'd get `underfull hbox'es.

\doruler{\8\8\8}3{in}
\endexample
\enddesc

\begindesc
\cts vtop {\<mat\'eriel en mode vertical>}
\aux\cts vtop {{\bt to} \<dimension> \rqbraces{\<mat\'eriel en mode vertical>}}
\aux\cts vtop {{\bt spread} \<dimension> \rqbraces{\<mat\'eriel en mode vertical>}}
\cts vbox {\rqbraces{\<mat\'eriel en mode vertical>}}
\aux\cts vbox {{\bt to} \<dimension> \rqbraces{\<mat\'eriel en mode vertical>}}
\aux\cts vbox {{\bt spread} \<dimension> \rqbraces{\<mat\'eriel en mode vertical>}}
\explain
^^{hbox//construction avec \b\tt\\hbox\e}
Ces commandes
produisent une \minref{vbox} (\minref{bo\^\i te} verticale)
contenant du \<ma\-t\'eriel en mode vertical>.
Les accolades autour du \<mat\'eriel en mode vertical> d\'efinissent un groupe.
\TeX\ est en mode vertical interne quand il assemble la bo\^\i te.
\TeX\ ne change pas la taille de la bo\^\i te une fois qu'elle est produite.

La diff\'erence entre |\vtop| et |\vbox| est o\`u \TeX\ met
le point de r\'ef\'erence de la vbox construite.
Normalement, le point de r\'ef\'erence obtenu de |\vtop| tend \`a \^etre sur ou pr\`es
du haut de la vbox construite,
tandis que
le point de r\'ef\'erence obtenu de |\vbox| tend \`a \^etre sur ou pr\`es
du bas de la vbox construite.
Ainsi une rang\'ee de vbox toutes construites avec |\vtop|
tendra \`a avoir leurs hauts \`a peu pr\`es align\'es,
tandis qu'une rang\'e de vbox toutes construites avec |\vbox|
tendra \`a avoir leurs bas \`a peu pr\`es align\'es.

|\vtop| et |\vbox| sont souvent pratiques
quand vous voulez garder du texte ensemble sur une seule page.
(Pour cela, peu importe normalement quelle commande vous utilisez.)
Si votre usage de ces commandes
emp\^eche \TeX\ de couper des pages d'une mani\`ere acceptable,
\TeX\ se plaindra qu'il a trouv\'e un ``overfull'' ou un ``underfull vbox while
|\output| is active''.

La hauteur d'une vbox d\'epend des 
arguments de |\vtop| ou |\vbox|.
Pour |\vbox|, \TeX\ d\'etermine la hauteur de la mani\`ere suivante~:
\ulist\compact
\li Si vous ne sp\'ecifiez que du \<mat\'eriel en mode vertical>,
la vbox aura sa hauteur naturelle.
\li Si vous sp\'ecifiez |to| \<dimension>,
la hauteur de la vbox sera fix\'ee \`a \<dimension>.
\li Si vous sp\'ecifiez |spread| \<dimension>, la hauteur de la vbox sera
sa hauteur naturelle plus \<dimension>, c'est-\`a-dire,
que la hauteur de la vbox sera \'etir\'ee verticalement de \<dimension>.
\endulist
\noindent
Pour |\vtop|,
\TeX\ construit la bo\^\i te en utilisant ses r\`egles pour |\vbox| et ensuite
r\'epartit l'extension verticale entre la hauteur et la profondeur comme
d\'ecrit ci-dessous.

Normalement, la largeur d'une vbox construite est la largeur de l'\'el\'e\-ment le plus
large contenu\footnote
{Plus pr\'ecis\'ement, c'est la distance du point de r\'ef\'erence au cot\'e le plus \`a
droite de la vbox construite.  Par cons\'equent,
si vous d\'eplacez un des \'el\'ements vers la droite en utilisant ^|\moveright| ou
^|\moveleft| (avec une distance n\'egative),
la vbox construite pourra \^etre plus large.}.
Les r\`egles de r\'epartitions de l'extension verticale entre la
hauteur et la profondeur sont plus compliqu\'ees~:
\ulist
\li Pour |\vtop|,
la hauteur est la hauteur de son premier \'el\'ement, si cet \'el\'ement est une bo\^\i te ou un filet.
Autrement la hauteur est \`a z\'ero. La profondeur est toute extension verticale
restante quand la hauteur a \'et\'e enlev\'ee.
\li Pour |\vbox|,
la profondeur est la profondeur de son dernier \'el\'ement, si cet \'el\'ement est une bo\^\i te ou un filet.
Autrement la profondeur est \`a z\'ero. La hauteur est toute extension verticale
restante quand la profondeur a \'et\'e enlev\'ee%
\footnote{En fait, il y a une complication suppl\'ementaire.
Supposez qu'apr\`es que la profondeur ait \'et\'e d\'etermin\'ee
en utilisant les deux r\`egles pr\'ec\'edentes, la profondeur devienne plus grande que
^|\boxmaxdepth|.
Alors la profondeur est r\'eduite \`a |\boxmaxdepth| et la hauteur est ajust\'ee
en cons\'equence.}.
\endulist

La commande |\vfil| (\xref\vfil) est pratique pour remplir une vbox
^^|\vfil//remplir une vbox|
avec de l'espace vide quand le mat\'eriel dans la bo\^\i te n'est pas aussi grand que
l'extention verticale de la bo\^\i te.
\example
\hbox{\hsize = 10pc \raggedright\parindent = 1em
\vtop{In this example, we see how to use vboxes to
produce the effect of double columns.  Each vbox
contains two paragraphs, typeset according to \TeX's
usual rules except that it's ragged right.\par
This isn't really the best way to get true double
columns because the columns}
\hskip 2pc
\vtop{\noindent
aren't balanced and we haven't done anything to choose
the column break automatically or even to fix up the
last line of the first column.\par
However, the technique of putting running text into a
vbox is very useful for placing that text where you
want it on the page.}}
|
\produces
\hbox{\hsize = 10pc \raggedright\parindent = 1em
\vtop{In this example, we see how to use vboxes to
produce the effect of double columns.  Each vbox
contains two paragraphs, typeset according to \TeX's
usual rules except that it's ragged right.\par

This isn't really the best way to get true double
columns because the columns}
\hskip 2pc
\vtop{\noindent
aren't balanced and we haven't done anything to choose
the column break automatically or even to fix up the
last line of the first column.\par

However, the technique of putting running text into a
vbox is very useful for placing that text where you
want it on the page.}}
\nextexample
\hbox{\hsize = 1in \raggedright\parindent = 0pt
\vtop to .75in{\hrule This box is .75in deep. \vfil\hrule}
\qquad
\vtop{\hrule This box is at its natural depth. \vfil\hrule}
\qquad
\vtop spread .2in{\hrule This box is .2in deeper than
                  its natural depth.\vfil\hrule}}
|
\produces
\hbox{\hsize = 1in \raggedright\parindent=0pt
\vtop to .75in{\hrule This box is .75in deep. \vfil\hrule}
\qquad
\vtop{\hrule This box is at its natural depth. \vfil\hrule}
\qquad
\vtop spread .2in{\hrule This box is .2in deeper than
                  its natural depth.\vfil\hrule}}
\nextexample
% See how \vbox lines up boxes at their bottoms
% instead of at their tops.
\hbox{\hsize = 1in \raggedright
\vbox to .5in{\hrule This box is .5in deep.\vfil\hrule}
\qquad
\vbox to .75in{\hrule This box is .75in deep.\vfil\hrule}}
|
\produces
\hbox{\hsize = 1in \raggedright
\vbox to .5in{\hrule This box is .5in deep.\vfil\hrule}
\qquad
\vbox to .75in{\hrule This box is .75in deep.\vfil\hrule}}
\vskip 16pt % to avoid running into the next command description
\endexample
\enddesc

\begindesc
\margin{Two commands interchanged here.}
\cts boxmaxdepth {\param{dimension}}
\explain
Ce param\`etre contient une dimension $D$.
\TeX\ ne construira pas une bo\^\i te dont la profondeur d\'epassera $D$.
Si vous produisez une bo\^\i te dont la profondeur $d$ doit d\'epasser $D$,
\TeX\ transf\`erera la
profondeur exc\'edentaire dans la hauteur de la bo\^\i te, en abaissant en r\'ealit\'e le 
\minref{point de r\'ef\'erence} de la bo\^\i te de $d-D$.
Si vous mettez |\boxmaxdepth| \`a z\'ero, \TeX\ alignera une rang\'ee de vbox
de telle mani\`ere que leurs fronti\`eres du bas seront toutes sur la m\^eme ligne horizontale.
\PlainTeX\ met |\boxmaxdepth| \`a |\maxdimen| \ctsref{\maxdimen},
donc |\boxmaxdepth| n'affectera pas vos bo\^\i tes \`a moins que vous le changiez.
\enddesc

\begindesc
\cts underbar {\<argument>}
\explain
Cette commande met \<argument> dans une \minref{hbox}
et la souligne sans faire attention \`a tout ce que d\'epasse sous la
\minref{ligne de base} de la bo\^\i te.
\example
\underbar{Why not learn \TeX?}
|
\produces
\underbar{Why not learn \TeX?}
\endexample
\enddesc

\begindesc
\cts everyhbox {\param{liste de token}}
\cts everyvbox {\param{liste de token}}
\explain
Ces param\`etres contiennent des liste de token que \TeX\ d\'eveloppe au d\'ebut de
toutes \minref{hbox} ou \minref{vbox} qu'il construit.
Tout \'el\'ement r\'esultant de l'expansion devient alors
le d\'ebut de la liste d'\'el\'ements pour la bo\^\i te.
Par d\'efaut, ces listes de token sont vides.
\enddesc

%==========================================================================
\subsection {Remplir et r\'ecuperer le contenu de bo\^\i tes}

\begindesc
\bix^^{registres de \boite}
%
\cts setbox {\<registre>\thinspace{\bt =}\thinspace\<bo\^\i te>}
\cts box {\<registre>}
\explain
^^{assignements//de \boites}
Ces commandes remplissent et r\'ecup\`erent respectivement le contenu
du registre de bo\^\i te
dont le num\'ero est \<registre>.  
Notez que vous remplissez un registre de bo\^\i te d'une mani\`ere un peu diff\'erente de celle
utilis\'ee pour les autres types de registres~: vous utilisez 
|\setbox|$\,n$~|=| plut\^ot que |\box|$\,n$~|=|.

\emph{R\'ecup\'erer le contenu d'un registre de bo\^\i te
avec ces commandes a l'effet secondaire d'enlever ce qu'il contient,
donc le registre de bo\^\i te devient vide.}  Si vous ne voulez pas que cela arrive, 
vous pouvez utiliser |\copy| (voir ci-dessous) pour r\'ecup\'erer le contenu.
Vous devez utiliser |\box| de pr\'ef\'erence \`a |\copy|
quand vous n'avez plus besoin du contenu d'un registre de bo\^\i te apr\`es l'avoir utilis\'e,
de mani\`ere \`a ne pas augmenter la m\'emoire de \TeX\ en la remplissant avec des bo\^\i tes obsol\`etes.
\example
\setbox0 = \hbox{mushroom}
\setbox1 = \vbox{\copy0\box0\box0}
\box1
|
\produces
\setbox0 = \hbox{mushroom}
\setbox1 = \vbox{\copy0\box0\box0}
\box1
\endexample
\enddesc

\begindesc
\cts copy {\<registre>}
\explain
^^{\boites//copie}\minrefs{box}
Cette commande
produit une copie d'un registre de bo\^\i te \<registre>.
Cette commande est utile quand vous voulez r\'ecup\'erer le contenu d'un registre
de bo\^\i te mais ne voulez pas en d\'etruire le contenu.
(R\'ecup\'erer le contenu d'un registre avec ^|\box| rend le registre vide.)
\example
\setbox0 = \hbox{good }
Have a \copy0 \box0 \box0 day!!
|
\produces
\setbox0 = \hbox{good }
Have a \copy0 \box0 \box0 day!
\endexample
\enddesc

\begindesc
\cts unhbox {\<registre>}
\cts unvbox {\<registre>}
^^{\boites//extraction de contenu de}
\explain
Ces commandes produisent la liste contenue dans un registre de bo\^\i te \<registre> et rend
ce registre de bo\^\i te vide.
|\unhbox| s'applique aux registres de bo\^\i te contenant 
des hbox et |\unvbox| s'applique aux registres de bo\^\i te contenant des vbox.
Vous devez utiliser ces commandes en pr\'ef\'erence \`a |\unhcopy| et |\unvcopy|
(voir ci-dessous)
quand vous n'avez pas besoin de ce qu'il y a dans le registre de bo\^\i te apr\`es l'avoir utilis\'e,
de mani\`ere \`a ne pas augmenter la m\'emoire de \TeX\ en la remplissant avec des bo\^\i tes obsol\`etes.
\example
\setbox0=\hbox{The Mock Turtle sighed deeply, and
drew the back of one flapper across his eyes. }
\setbox1=\hbox{He tried to speak, but sobs choked 
               his voice. }
\unhbox0 \unhbox1
% \box0 \box1 would set two hboxes side by side
% (and produce a badly overfull line).
\box1 % produces nothing
|
\produces
\setbox0=\hbox{The Mock Turtle sighed deeply, and
drew the back of one flapper across his eyes. }
\setbox1=\hbox{He tried to speak, but sobs choked 
               his voice. }
\unhbox0 \unhbox1
% \hbox0 \hbox1 would set two hboxes side by side
% (and produce a badly overfull line).
\box1 % Produces nothing.
\endexample
\enddesc

\begindesc
\cts unhcopy {\<registre>}
\cts unvcopy {\<registre>}
^^{\boites//extraction du contenu de}
^^{\boites//copie}\minrefs{box}
\explain
Ces commandes produisent la liste contenue dans un registre de bo\^\i te \<registre> et 
sans d\'eranger le contenu des registres.
|\unhbox| s'applique aux registres de bo\^\i te contenant 
des hbox et |\unvbox| s'applique aux registres de bo\^\i te contenant des vbox.
\example
\setbox0=\hbox{The Mock Turtle sighed deeply, and
drew the back of one flapper across his eyes. }
\setbox1=\hbox{He tried to speak, but sobs choked his 
          voice. }
\unhcopy0 \unhcopy1\par\noindent
% \box0 \box1 would set two hboxes side by side
% (and produce a badly overfull line).
\box1 % Produces an hbox (which can't be broken).
|
\produces
\setbox0=\hbox{The Mock Turtle sighed deeply, and
drew the back of one flapper across his eyes. }
\setbox1=\hbox{He tried to speak, but sobs choked his 
          voice. }
\unhcopy0 \unhcopy1\par\noindent
% \hbox0 \hbox1 would set two hboxes side by side
% (and produce a badly overfull line).
\box1 % produces an hbox (which can't be broken)
\endexample
\enddesc

\see |\wd|, |\dp|, |\ht| (\xref\ht).
\eix^^{registres de \boite}

%==========================================================================
\subsection {D\'eplacer des bo\^\i tes}

\begindesc
\cts moveleft {\<dimension> \<bo\^\i te>}
\cts moveright {\<dimension> \<bo\^\i te>}
\bix^^{\boites//d\'eplacement}
\minrefs{box}
\explain
Ces commandes d\'eplacent \<bo\^\i te> \`a gauche ou \`a droite de \<dimension> (qui peut \^etre
n\'egatif). Vous ne pouvez appliquer |\moveleft| et |\moveright| qu'\`a des bo\^\i tes
qui sont dans une \minref{liste verticale}.
\example
\vbox{\vbox{Phoebe}\vbox{walked}%
\moveleft 20pt\vbox{a}\moveright 20pt\vbox{crooked}%
\vbox{mile.}}
|
\produces
\hfuzz = \maxdimen
\vbox{\vbox{Phoebe}\vbox{walked}%
\moveleft 20pt\vbox{a}\moveright 20pt\vbox{crooked}%
\vbox{mile.}}
\endexample
\enddesc

\begindesc
\cts lower {\<dimension> \<bo\^\i te>}
\cts raise {\<dimension> \<bo\^\i te>}
\explain
Ces commandes d\'eplacent \<bo\^\i te> en haut ou en bas de \<dimension> 
(qui peut \^etre n\'egatif). 
Vous ne pouvez appliquer |\moveleft| et |\moveright| qu'\`a des bo\^\i tes
qui sont dans une \minref{liste horizontale}.
\example
Are you feeling \lower 6pt \hbox{depressed} about the
                \raise 6pt \hbox{bump} on your nose?
|
\produces
Are you feeling \lower 6pt \hbox{depressed} about the
                \raise 6pt \hbox{bump} on your nose?
\endexample

\eix^^{\boites//d\'eplacement}
\enddesc

%==========================================================================
\subsection {Dimensions des registres de bo\^\i tes}

\begindesc
\bix^^{registres de \boite}
\bix^^{dimensions//de registres de \boite}
\cts ht {\<registre>\param{dimension}}
\cts dp {\<registre>\param{dimension}}
\cts wd {\<registre>\param{dimension}}
\explain
Ces param\`etres font r\'ef\'erence \`a la ^{hauteur}, la ^{profondeur} et la ^{largeur}
respectivement du \minref{registre} de \minref{bo\^\i te} \<registre>.
Vous pouvez les utiliser pour conna\^\i tre les dimensions d'une bo\^\i te.
Vous pouvez aussi changer les dimensions d'une bo\^\i te, mais ce n'est pas une mince affaire~;
si vous voulez \^etre aventureux vous pouvez
apprendre tout sur cela dans les \
{pages~388--389}{448--449}.
\example
\setbox0 = \vtop{\hbox{a}\hbox{beige}\hbox{bunny}}%
The box has width \the\wd0, height \the\ht0,
and depth \the\dp0.
|
\produces
\setbox0 = \vtop{\hbox{a}\hbox{beige}\hbox{bunny}}%
The box has width \the\wd0, height \the\ht0,
and depth \the\dp0.
\endexample
\eix^^{registres de \boite}
\eix^^{dimensions//de registres de \boite}
\enddesc

%==========================================================================
\subsection {Struts, fantomes et bo\^\i tes vides}

\begindesc
\margin{This subsection is a merger of two previous subsections.
The commands have also been reordered.}
\bix^^{struts}
\cts strut {}
\explain
Cette commande produit une bo\^\i te dont la largeur est \`a z\'ero
et dont la hauteur (|8.5pt|) et la profondeur (|3.5pt|)
sont celles d'une ligne de caract\`ere plus ou moins en |cmr10|,
la police par d\'efaut de \plainTeX.
Son utilisation principale est de forcer des lignes \`a avoir la m\^eme hauteur
quand vous avez d\'ebranch\'e le ressort inter-ligne de \TeX\ avec
|\offinter!-lineskip| ^^|\offinterlineskip| ou une commande similaire,
par exemple, quand vous construisez un alignement.
Si la hauteur naturelle d'une ligne est
trop petite, vous pouvez la ramener au standard en incluant un |\strut|
dans la ligne.  Le strut forcera la hauteur et la profondeur de la ligne
\`a \^etre plus grande, mais il n'imprime rien ni ne consomme aucun espace
horizontal.

Si vous mettez des caract\`eres d'une police qui est plus grande ou plus petite que |cmr10|,
vous devez red\'efinir |\strut| pour ce contexte.
\example
\noindent % So we're in horizontal mode.
\offinterlineskip % So we get the inherent spacing.
% The periods in this vbox are not vertically equidistant.
\vtop{\hbox{.}\hbox{.(}\hbox{.x}
   \hbox{.\vrule height 4pt depth 0pt}}\qquad
% The periods in this vbox are vertically equidistant
% because of the struts.
\vtop{\hbox{.\strut}\hbox{.(\strut}\hbox{.x\strut}
   \hbox{.\vrule height 4pt depth 0pt\strut}}
|
\produces
\noindent % So we're in horizontal mode.
\offinterlineskip % So we get the inherent spacing.
% The periods in this vbox are not vertically equidistant.
\vtop{\hbox{.}\hbox{.(}\hbox{.x}
   \hbox{.\vrule height 4pt depth 0pt}}\qquad
% The periods in this vbox are vertically equidistant
% because of the struts.
\vtop{\hbox{.\strut}\hbox{.(\strut}\hbox{.x\strut}
   \hbox{.\vrule height 4pt depth 0pt\strut}}
\endexample
\enddesc

\begindesc
\cts mathstrut {}
\explain
Cette commande produit une formule fant\^ome dont la largeur est \`a z\'ero
et dont la largeur et la hauteur sont les m\^emes que celle d'une parenth\`ese gauche.
|\mathstrut| est en fait d\'efini comme `|\vphantom(|'.
Son utilisation principale est d'obtenir des radicaux, soulign\'es et surlign\'es align\'es
avec d'autres radicaux, soulign\'es et surlign\'es d'une formule.
C'est un peu comme ^|\strut| (\xref \strut),
sauf qu'il s'ajuste lui-m\^eme aux diff\'erents \minref{style}s
qui peuvent appara\^\i tre dans une formule math\'ematique.  
\example
$$\displaylines{
\overline{a_1a_2} \land \overline{b_1b_2} 
\quad{\rm versus}\quad \overline{a_1a_2\mathstrut}
  \land \overline{b_1b_2\mathstrut}\cr
\sqrt{\epsilon} + \sqrt{\xi} \quad{\rm versus}\quad
\sqrt{\epsilon\mathstrut} + \sqrt{\xi\mathstrut}\cr}$$
|
\dproduces
\kern 4pt
$$\displaylines{
\overline{a_1a_2} \land \overline{b_1b_2} 
\quad{\rm versus}\quad \overline{a_1a_2\mathstrut}
  \land \overline{b_1b_2\mathstrut}\cr
\sqrt{\epsilon} + \sqrt{\xi} \quad{\rm versus}\quad
\sqrt{\epsilon\mathstrut} + \sqrt{\xi\mathstrut}\cr}$$
\endexample
\eix^^{struts}
\enddesc

\begindesc
\cts phantom {\<argument>}
\explain
\bix^^{\boites//fant\^ome}
\bix^^{fant\^omes}
Cette commande produit une \minref{bo\^\i te} vide
ayant la m\^eme taille et placement
que \<argument> aurait s'il \'etait compos\'e.  
Une utilisation de |\phantom| est de r\'eserver de l'espace pour un symbole qui pour certaines
raisons demande \`a \^etre dessin\'e \`a la main.
\example
$1\phantom{9}2$
|
\produces
$1\phantom{9}2$
\endexample
\enddesc

\begindesc
\cts hphantom {\<argument>}
\cts vphantom {\<argument>}
\explain
Ces commandes produisent des bo\^\i tes\minrefs{bo\^\i te}\ fant\^ome qui
n'impriment rien~:
\ulist\compact
\li |\hphantom| produit une bo\^\i te avec la m\^eme largeur que \<argument>
mais de hauteur et profondeur z\'ero.
\li |\vphantom| produit une bo\^\i te avec les m\^emes hauteur et profondeur que \<argu\-ment>
mais de largeur z\'ero.
\endulist
\noindent
Leur usage principal est de forcer une sous-formule \`a avoir un dimension
horizontale ou verticale minimum.
\example
$$\left[\vphantom{u\over v}t\right] \star
      \left[{u\over v}\right]\quad
   \{\hphantom{xx}\}$$
|
\dproduces
\kern 4pt
$$\left[\vphantom{u\over v}t\right] \star
      \left[{u\over v}\right]\quad
   \{\hphantom{xx}\}$$
\endexample
\enddesc

\begindesc
\cts smash {\<argument>}
\explain
Cette commande compose \<argument>,
mais force la hauteur et la profondeur de sa \minref{bo\^\i te} contenante \`a z\'ero.
Vous pouvez utiliser |\smash| et |\vphantom| en combinaison pour donner \`a une sous-formule
toutes hauteurs et profondeurs que vous d\'esirez.
\example
$${\smash{r_m \brace r_n}\vphantom{r}} \Longrightarrow r$$
|
\dproduces
\kern 4pt
$${\smash{r_m \brace r_n}\vphantom{r}} \Longrightarrow r$$
\endexample
\eix^^{\boites//fant\^ome}
\eix^^{fant\^omes}
\enddesc

\begindesc
^^{\boites//vides}
\cts null {}
\explain
Cette commande produit une \minref{hbox} vide.
\example
\setbox0 = \null
The null box \null has width \the\wd0, height \the\ht0,
and depth \the\dp0.
|
\produces
\setbox0 = \null
The null box \null has width \the\wd0, height \the\ht0,
and depth \the\dp0.
\endexample
\enddesc


%==========================================================================
\subsection {Param\`etres faisant partie des bo\^\i tes mal form\'ees}

\begindesc
\bix^^{\boites//overfull}
\bix^^{\boites//underfull}
\cts overfullrule {\param{dimension}}
\explain
Ce param\`etre sp\'ecifie la largeur du filet 
que \TeX\ positionne pour un ``overfull \minref{hbox}''.
\PlainTeX\ le met \`a |5pt|.
\enddesc

\begindesc
\cts hbadness {\param{nombre}}
\cts vbadness {\param{nombre}}
\explain
Ces param\`etres sp\'ecifient les seuils de \minref{badness} horizontal 
et vertical pour rapporter des ``underfull'' ou ``overfull boxes''
|\hbadness| s'applique aux hbox et |\vbadness| s'applique aux vbox.
Si la m\'ediocrit\'e d'une bo\^\i te construite d\'epasse le seuil,
\TeX\ reportera une erreur.  Si vous augmentez les seuils
(les valeurs par d\'efaut de \plainTeX\ sont de $1000$),
\TeX\ sera moins amen\'e \`a se plaindre.
Notez que les valeurs de |\hbadness| et |\vbadness| n'ont pas d'effet sur
l'apparence de votre document compos\'e~; ils n'affectent que le message d'erreur
que vous obtenez.
Voir la \knuth{page~302}{353} pour une description pr\'ecise 
sur comment \TeX\ d\'ecide quand se plaindre d'``overfull'' ou d'``underfull
box''.

\margin{The material on tolerance was inappropriate here and has been
removed.  A new example replaces it.}
\example
\hbadness = 10000 % Suppress any hbadness complaints.
\hbox to 2in{a b}\par
\hbadness = 500 % Report hbadness exceeding 500.
\hbox to 2in{a\hskip 0pt plus .5in b}
|
\logproduces
Underfull \hbox (badness 5091) detected at line 4
\tenrm a b

\hbox(6.94444+0.0)x144.54, glue set 3.70787
.\tenrm a
.\glue 0.0 plus 36.135
.\tenrm b
|
\endexample
\enddesc

\begindesc
\cts badness {}
\explain
Cette commande donne la valeur num\'erique de la m\'ediocrit\'e de la \minref{bo\^\i te}
(aussi bien horizontale que verticale) que \TeX\ a le plus r\'ecemment produit.
Si la bo\^\i te \'etait ``overfull'', |\badness| sera de $1000000$~; dans tous les autres cas
il sera entre $0$ et $10000$.
\enddesc

\begindesc
\cts hfuzz {\param{dimension}}
\cts vfuzz {\param{dimension}}
\explain
Ces param\`etres sp\'ecifient les valeurs dont une
\minref{bo\^\i te} peut d\'epasser sa taille naturelle avant que \TeX\ la consid\`ere comme
\'etant overfull. ^^{\boites//overfull}
|\hfuzz| s'applique aux hbox et |\vfuzz| s'applique aux vbox.
\PlainTeX\ met ces deux param\`etres \`a |0.1pt|.
\example
\hfuzz = .5in
\hbox to 2in{This box is longer than two inches.}
% No error results
|
\produces
\hfuzz = .5in
\hbox to 2in{This box is longer than two inches.}
% No error results
\doruler{\8\8\8}3{in}
\endexample

\eix^^{\boites//overfull}
\eix^^{\boites//underfull}
\eix^^{commandes de \boites}
\enddesc

\see |\tolerance| \ctsref{\tolerance}.
%==========================================================================
\section {Retrouver le dernier \'el\'ement d'une liste}

\begindesc
^^{\boites//derni\`ere \boite\ dans une liste}
^^{cr\'enages//dernier cr\'enage dans une liste}
^^{p\'enalit\'es//derni\`ere p\'enalit\'e dans une liste}
^^{ressort//dernier \'el\'ement de ressort dans une liste}
\margin{This section has been moved in its entirety from the chapter
(`Commands for general operations')}
\cts lastkern {}
\cts lastskip {}
\cts lastpenalty {}
\cts lastbox {}
\explain
Ces s\'equences de contr\^ole donnent la valeur du dernier \'el\'ement de la liste
courante.  Ce ne sont pas de vraies commandes parce qu'elles ne peuvent appara\^\i tre que
comme partie d'un argument.
Si le dernier \'el\'ement de la liste n'est pas du type indiqu\'e,
elles donnent une valeur z\'ero (ou une bo\^\i te vide, dans le cas de |\lastbox|).
Par exemple,
si le dernier \'el\'ement de la liste courante est un \minref{cr\'enage},
|\lastkern| donne la dimension de ce cr\'enage~; si ce
n'est pas un cr\'enage, il donne une dimension de $0$.

Utiliser |\lastbox| a l'effet suppl\'ementaire d'enlever la derni\`ere bo\^\i te de
la liste.
Si vous voulez que le |\last!-box| original reste dans la liste, vous
devez en ajouter une copie dans la liste.
|\last!-box| n'est pas permise dans une liste math\'ematique ou dans la liste
verticale~principale.

Ces s\'equences de contr\^ole sont
plus utiles apr\`es des appels de macro qui peuvent avoir ins\'er\'e des entit\'es 
du genre indiqu\'e. 

\example
\def\a{two\kern 15pt}
one \a\a\hskip 2\lastkern three\par
% Get three times as much space before `three'.
\def\a{\hbox{two}}
one \a
\setbox0 = \lastbox % Removes `two'.
three \box0.
|
\produces
\def\a{two\kern 15pt}
one \a\a\hskip 2\lastkern three\par
% get three times as much space before `three'
\def\a{\hbox{two}}
one \a
\setbox0 = \lastbox % removes `two'
three \box0.
\endexample
\enddesc

\begindesc
\cts unkern {}
\cts unskip {}
\cts unpenalty {}
\explain
Si le dernier \'el\'ement de la liste courante est de type \minref{cr\'enage},
\minref{ressort} ou \minref{p\'enalit\'e} respectivement, ces commandes l'enl\`event
de cette liste.  Si l'\'el\'ement n'est pas du bon type, ces commandes n'ont
pas d'effet.  Comme ^|\lastbox|, vous ne pouvez pas les appliquer \`a des listes en 
mode math\'ema\-tique ou \`a la liste verticale principale.
Ces commandes
sont plus pratiques apr\`es un appel de macro connu pour avoir ins\'er\'e
un \'el\'ement sp\'ecifique que vous ne voulez pas ici.
\TeX\ ne procure pas de commande ^|\unbox| parce que |\lastbox| produit
\`a peu pr\`es le m\^eme effet.
\enddesc


%==========================================================================
\section {filets et r\`eglures}

\begindesc
\bix^^{filets}
\bix^^{filets horizontaux}
\bix^^{filets verticaux}
%
\cts hrule {}
\aux\cts hrule {\bt height \<dimension> width \<dimension> depth \<dimension>}
\cts vrule {}
\aux\cts vrule {{\bt width \<dimension> height \<dimension> depth \<dimension>}}
\explain
La commande |\hrule| produit un filet horizontal. La commande |\vrule| 
produit un filet vertical.  Vous pouvez sp\'ecifier tout ou une partie des largeur,
hauteur et profondeur du filet---\TeX\ apporte des valeurs par d\'efaut pour celles
que vous omettez.  Vous pouvez donner les dimensions du filet dans n'importe quel ordre~;
les formes list\'ees ci-dessus vous montrent simplement deux combinaisons possibles.  Vous
pouvez m\^eme donner une dimension d'un type donn\'e plus d'une fois---si vous le faites,
la derni\`ere sera celle qui sera prise en compte.

Si vous ne sp\'ecifiez pas la largeur d'un filet horizontal, le filet est
\'etendu horizontalement jusqu'aux limites de la \minref{bo\^\i te} ou de
l'\minref{alignement} le plus interne qui contient le filet.  Si vous ne sp\'ecifiez pas
la hauteur d'un filet horizontal, elle sera par d\'efaut de |0.4pt|~; si vous ne
sp\'ecifiez pas la profondeur d'un filet horizontal, elle sera par d\'efaut de |0pt|.

Si vous ne sp\'ecifiez pas la largeur d'un filet vertical, elle sera par d\'efaut de
|0.4pt|. Si vous ne sp\'ecifiez pas la hauteur d'un filet vertical, 
le filet est \'etendu jusqu'aux limites de la \minref{bo\^\i te} ou de
l'\minref{alignement} le plus interne qui contient le filet.

\TeX\ traite un filet horizontal comme un \'el\'ement naturellement vertical
et un filet vertical comme un \'el\'ement naturellement horizontal.
Donc un filet horizontal n'est l\'egal qu'en \minref{mode vertical},
Tandis qu'un filet vertical n'est l\'egal qu'en \minref{mode horizontal}.
^^{mode horizontal//filets en}
^^{mode vertical//filets en}
Si cela semble surprenant, visualisez le---un filet horizontal va
de gauche \`a droite et s\'epare des \'el\'ements verticaux dans une s\'equence, tandis
qu'un filet vertical va de haut en bas et
s\'epare des \'el\'ements horizontaux dans une s\'equence.

\example
\hrule\smallskip
\hrule width 2in \smallskip
\hrule width 3in height 2pt \smallskip
\hrule width 3in depth 2pt
|
\produces
\medskip
\hrule\smallskip
\hrule width 2in \smallskip
\hrule width 3in height 2pt \smallskip
\hrule width 3in depth 2pt
\nextexample
% Here you can see how the baseline relates to the
% height and depth of an \hrule.
\leftline{
   \vbox{\hrule width .6in height 5pt depth 0pt}
   \vbox{\hrule width .6in height 0pt depth 8pt}
   \vbox{\hrule width .6in height 5pt depth 8pt}
   \vbox{\hbox{ baseline}\kern 3pt \hrule width .6in}
}
|
\produces
\medskip
\leftline{
   \vbox{\hrule width .6in height 5pt depth 0pt}
   \vbox{\hrule width .6in height 0pt depth 8pt}
   \vbox{\hrule width .6in height 5pt depth 8pt}
   \vbox{\hbox{ baseline}\kern 3pt \hrule width .6in}
}
\nextexample
\hbox{( {\vrule} {\vrule width 8pt} )}
\hbox {( {\vrule height 13pt depth 0pt}
   {\vrule height 13pt depth 7pt} x)}
% the parentheses define the height and depth of each of the 
% two preceding boxes; the `x' sits on the baseline
|
\produces
\medskip
\hbox{( {\vrule} {\vrule width 8pt} )}
\hbox {( {\vrule height 13pt depth 0pt}
   {\vrule height 13pt depth 7pt} x)}
\endexample

\eix^^{filets}
\eix^^{filets horizontaux}
\eix^^{filets verticaux}
\enddesc

\begindesc
\bix^^{r\`eglures}
\easy\cts leaders {\<bo\^\i te ou filet> \<commande de saut>}
\cts cleaders {\<bo\^\i te ou filet> \<commande de saut>}
\cts xleaders {\<bo\^\i te ou filet> \<commande de saut>}
\explain
Ces commandes produisent des \minref{r\'eglures}, c'est-\`a-dire, qu'elles remplissent un
espace horizontal ou vertical avec des
copies d'un motif \seeconcept{filets}.
La \<bo\^\i te> ou le \<filet> sp\'ecifie une r\'eglure,
c'est-\`a-dire, une seule copie du motif,
tandis que la \<commande de saut> sp\'ecifie une 
fen\^etre devant \^etre remplie avec une
rang\'ee ou une colonne de r\'eglures.
Le motif est r\'ep\'et\'e autant de fois qu'il peut rentrer dans la fen\^etre.
Si \<commande de saut> est un saut horizontal, la fen\^etre 
contient une rang\'ee de r\'eglures et \TeX\ doit \^etre en mode horizontal~;
si \<commande de saut> est un saut vertical, la fen\^etre 
contient une colonne de r\'eglures et \TeX\ doit \^etre en mode vertical.

Les commandes diff\`erent dans leur fa\c con d'arranger le motif r\'ep\'et\'e 
et o\`u elles mettent l'espace restant~:

\ulist\compact
\li Pour |\leaders|, \TeX\ aligne une rang\'ee de r\'eglures avec la limite gauche de
la \minref{bo\^\i te} $B$ la plus interne qui contient le r\'esultat de la
commande |\leaders|.
Elle aligne une colonne de r\'eglures avec le haut de $B$.
Ces r\'eglures qui tombent enti\`erement dans la fen\^etre sont gard\'ees.
Tout espace en plus en haut et en bas de la fen\^etre est laiss\'e vide.
\li Pour |\cleaders|, les r\'eglures sont centr\'ees dans la fen\^etre.
\li Pour |\xleaders| le motif est uniform\'ement distribu\'e \`a travers la fen\^etre.
Si l'espace en plus est $l$ et que la r\'eglure est r\'ep\'et\'ee $n$ fois,
\TeX\ met de l'espace de largeur ou hauteur $l/(n+1)$ entre des r\'eglures adjacentes et
aux deux bouts (gauche et droite ou haut et bas) des r\'eglures.
\endulist

\example
\def\pattern{\hbox to 15pt{\hfil.\hfil}}
\line{Down the Rabbit-Hole {\leaders\pattern\hfil} 1}
\line{The Pool of Tears {\leaders\pattern\hfil} 9}
\line{A Caucus-Race and a Long Tale {\cleaders\pattern
      \hfil} 19}
\line{Pig and Pepper {\xleaders\pattern\hfil} 27}
|
\produces
\def\pattern{\hbox to 15pt{\hfil.\hfil}}\par
\line{Down the Rabbit-Hole {\leaders\pattern\hfil} 1}
\line{The Pool of Tears {\leaders\pattern\hfil} 9}
\line{A Caucus-Race and a Long Tale {\cleaders\pattern
      \hfil} 19}
\line{Pig and Pepper {\xleaders\pattern\hfil} 27}
\nextexample
\def\bulletfill{\vbox to 3ex{\vfil\hbox{$\bullet$}\vfil}}%
\def\mybox{\vbox to 1in}
\def\myrule{\hrule width 4pt}\hsize=2in
\hrule \line{%
   \mybox{\myrule depth 8pt \leaders\bulletfill\vfill}
   \hfil
   \mybox{\myrule depth 15pt \leaders\bulletfill\vfill}
   \hfil
   \mybox{\myrule depth 18pt \cleaders\bulletfill\vfill}
   \hfil
   \mybox{\myrule depth 12pt \xleaders\bulletfill\vfill}%
}\hrule
|
\produces
\medskip
\def\bulletfill{\vbox to 3ex{\vfil\hbox{$\bullet$}\vfil}}%
\def\mybox{\vbox to 1in}\def\myrule{\hrule width 4pt}\hsize=2in
\hrule \line{%
   \mybox{\myrule depth 8pt \leaders\bulletfill\vfill}
   \hfil
   \mybox{\myrule depth 15pt \leaders\bulletfill\vfill}
   \hfil
   \mybox{\myrule depth 18pt \cleaders\bulletfill\vfill}
   \hfil
   \mybox{\myrule depth 12pt \xleaders\bulletfill\vfill}%
}\hrule
\endexample\enddesc

\begindesc
\cts dotfill {}
\cts hrulefill {}
\explain
^^{remplissage}
Ces commandes remplissent respectivement l'espace horizontal englob\'e avec 
une rang\'ee de points sur la ligne de base et avec
une ligne horizontale sur la ligne de base.
C'est habituellement une bonne id\'ee de laisser un espace entre 
|\dotfill| ou |\hrulefill|
et le texte qui la pr\'ec\`ede ou la suit (voir l'exemple ci-dessous).
\example
\hbox to 3in{Start {\dotfill} Finish}
\hbox to 3in{Swedish {\hrulefill} Finnish}
|
\produces
\par\hbox to 3in{Start {\dotfill} Finish}
\hbox to 3in{Swedish {\hrulefill} Finnish}
\endexample\enddesc

\begindesc
\cts leftarrowfill {}
\cts rightarrowfill {}
\explain
^^{remplissage}
Ces commandes remplissent l'espace horizontal englob\'e avec
des ^{fl\`eches} point\'ees vers la gauche ou vers la droite.
\example
\hbox to 3in{\vrule \rightarrowfill \ 3 in 
             \leftarrowfill\vrule}
|
\produces
\medskip
\hbox to 3in{\vrule \rightarrowfill \ 3 in 
        \leftarrowfill\vrule}
\endexample

\eix^^{r\`eglures}
\enddesc

%==========================================================================
\section {Alignements}

%==========================================================================
\subsection {Alignements tabul\'es}

\begindesc
\bix^^{alignements tabul\'es}
\bix^^{alignements//commandes pour}
%
\ctspecial + {{\bt \<texte>\thinspace\&\thinspace\<texte>%
   \thinspace\& $\cdots$ \\cr}} 
\ctsxrdef{@plus}
\cts tabalign {}
\explain
Ces commandes d\'ebutent une simple ligne dans un \minref{alignement} tabul\'e .
La seule diff\'erence entre |\+| et |\tabalign| est que 
|\+| est une macro outer ---vous ne pouvez pas l'utiliser quand \TeX\ lit des tokens \`a
haute vitesse \seeconcept{outer}.

Si vous placez un `|&|' \`a une position
\`a droite de toute tabulation existante dans un alignement tabul\'e,
le `|&|' \'etablit une nouvelle tabulation \`a cette position.
\example
\cleartabs % Nullify any previous \settabs.
\+ {\bf if }$a[i] < a[i+1]$ &{\bf then}&\cr
\+&&$a[i] := a[i+1]$;\cr
\+&&{\it found }$:=$ {\bf true};\cr
\+&{\bf else}\cr
\+&&{\it found }$:=$ {\bf false};\cr
\+&{\bf end if};\cr
|
\produces
\cleartabs % Nullify any previous \settabs.
\+ {\bf if }$a[i] < a[i+1]$ &{\bf then}&\cr
\+&&$a[i] := a[i+1]$;\cr
\+&&{\it found }$:=$ {\bf true};\cr
\+&{\bf else}\cr
\+&&{\it found }$:=$ {\bf false};\cr
\+&{\bf end if};\cr
\endexample
\enddesc

\begindesc
\cts settabs {\<nombre> {\bt \\columns}}
\aux\cts settabs {{\bt \\+} \<ligne d'exemple> {\bt \\cr}}
\explain
La premi\`ere forme de
cette commande d\'efinit un jeu de tabulations d'arr\^et ^^{tabulations}
pour un \minref{alignement} tabul\'e.
Elle demande \`a \TeX\ de mettre les tabulations d'arr\^et de mani\`ere \`a diviser chaque ligne en
\<nombre> parties \'egales.  \TeX\ prend la longueur d'une ligne comme \'etant
|\hsize|, normalement.  Vous pouvez rendre l'alignement plus \'etroit en diminuant
|\hsize|.
\margin{paragraph ``The tab settings $\ldots$'' moved to below.}
\example
{\hsize = 3in \settabs 3 \columns
\+$1$&one&first\cr
\+$2$&two&second\cr
\+$3$&three&third\cr}
|
\produces
{\hsize = 3in \settabs 3 \columns
\+$1$ & one & first\cr
\+$2$ & two & second\cr
\+$3$ & three & third\cr}

\noindent\doruler{\8\8\8}3{in}
\smallskip
\endexample

La seconde forme de cette commande d\'efinit des tabulations d'arr\^et en mettant les tabulations d'arr\^et
aux positions indiqu\'ees par les `|&| dans la ligne d'exemple.
La ligne d'exemple elle-m\^eme ne doit pas appara\^\i tre dans la sortie.  Quand vous
utilisez cette forme vous devrez normalement mettre du mat\'eriel dans la ligne d'exemple qui sera
un peu plus large que le plus large mat\'eriel correspondant dans l'alignement,
de mani\`ere \`a produire de l'espace entre les colonnes.
C'est ce que nous avons fait dans  
l'exemple ci-dessous.  Le mat\'eriel suivant la derni\`ere tabulation
d'arr\^et est inutile, puisqu'il n'a pas besoin de positionner
quoi que ce soit \`a l'endroit o\`u le |\cr|~appara\^\i t.

Les r\'eglages de tabulation \'etablies par |\settabs| restent actifs jusqu'\'a  
une nouvelle commande |\settabs| ou la fin du groupe contenant |\set!-tabs|. Ceci est vrai pour 
les deux formes de la commande.

\example
% The first line establishes the template.
\settabs \+$1$\qquad & three\quad & seventh\cr
\+$1$&one&first\cr
\+$2$&two&second\cr
\+$3$&three&third\cr
|
\produces
\settabs \+$1$\qquad & three\quad & seventh\cr % the sample line
\+$1$&one&first\cr
\+$2$&two&second\cr
\+$3$&three&third\cr
\smallskip
\endexample
\enddesc

\begindesc


\cts cleartabs {}
\explain
Cette commande efface toutes les tabulations \`a droite de la colonne courante.
Son utilisation principale est dans des applications telles que la composition de programme informatique
dans lesquels les positions de tabulation changent \`a chaque ligne.
\enddesc

\see |\cr|, |\endline|, |\crcr| (\xref \endline).
\eix^^{alignements tabul\'es}

%==========================================================================
\subsection {Alignement g\'en\'eraux}

\begindesc
{\tighten
\cts halign {{\bt \rqbraces{\<pr\'eambule> \\cr \<rang> \\cr $\ldots$ \<rang> \\cr}}}
   \xrdef{@and}\xrdef{@pound}
\aux\cts halign {{\bt to \<dimension>%
   \rqbraces{\<pr\'eambule> \\cr \<rang> \\cr $\ldots$ \<rang> \\cr}}}
\aux\cts halign {{\bt spread  \<dimension>%
   \rqbraces{\<pr\'eambule> \\cr \<rang> \\cr $\ldots$ \<rang> \\cr}}}
\par}
\explain
Cette commande produit un \minref{alignement} horizontal constitu\'e d'une
suite de rang\'ees, o\`u chaque rang\'ee \`a son tour contient une suite
d'entr\'ee de colonne. \TeX\ ajuste les largeurs des entr\'ees de colonne 
pour accommoder la plus large dans chaque colonne.

Un alignement horizontal n'appara\^\i t que quand \TeX\ est dans un \minref{mode} 
vertical. Nous vous recommandons d'apprendre les alignements en g\'en\'e\-ral
(\xref{alignement}) avant d'essayer d'utiliser la commande |\valign|.

Un alignement consiste en un ^{pr\'eambule}
suivi par le texte \`a aligner. Le pr\'eambule,
qui d\'ecrit la forme des rang\'ees qui le suivent, consiste en une
suite de patrons de colonne, s\'epar\'es par `|&|' et termin\'e
par |\cr|.
\bix^^{patron}
\bix^^{entr\'ee (colonne ou rang\'ee)}
Chaque rang\'ee consiste en une suite d'entr\'ees 
de ^{colonne}, s\'epar\'ees aussi par `|&|' et termin\'ees par |\cr|. Dans
un patron, `|#|' indique o\`u \TeX\ doit ins\'erer le
texte correspondant \`a l'entr\'ee de colonne. 
Par contre, |\settabs| utilise un patron fixe implicite \`a `|#|',
c'est-\`a-dire, qu'il ins\`ere simplement le texte tel quel.

\TeX\ compose chaque entr\'ee de colonne en mode horizontal restreint, 
c'est-\`a-dire, comme le contenu d'une \minref{hbox},
et englobe implicitement l'en\-tr\'ee dans un groupe.

La forme |to| de cette commande demande \`a \TeX\
d'\'etendre la largeur de l'alignement \`a \<dimension>,
en ajustant l'espace entre les colonnes.
La forme |spread| de cette commande demande \`a \TeX\
d'agrandir l'alignement de \<dimension> par rapport \`a sa largeur naturelle.
Ces formes sont comme les formes correspondantes de |\hbox| \ctsref\hbox.

Voir |\tabskip| \ctsref\tabskip{} pour un exemple d'utilisation de la
forme |to|.

\example
\tabskip = 1em \halign{%
   \hfil\it#\hfil&\hfil#\hfil&#&\hfil\$#\cr
   United States&Washington&dollar&1.00\cr
   France&Paris&franc&0.174\cr
   Israel&Jerusalem&shekel&0.507\cr
   Japan&Tokyo&yen&0.0829\cr}
|
\produces
\tabskip = 1em \halign{%
   \hfil\it#\hfil&\hfil#\hfil&#&\hfil\$#\cr
   United States&Washington&dollar&1.00\cr
   France&Paris&franc&0.174\cr
   Israel&Jerusalem&shekel&0.507\cr
   Japan&Tokyo&yen&0.0829\cr}
\endexample
\enddesc


\begindesc
{\tighten
\cts valign {{\bt \rqbraces{\<pr\'eambule>\\cr \<colonne>\\cr $\ldots$
   \<colonne>\\cr}}}
\aux\cts valign {{\bt to \<dimension>%
   \rqbraces{\<pr\'eambule>\\cr \<colonne>\\cr $\ldots$ \<colonne>\\cr}}}
\aux\cts valign {{\bt spread \<dimension>%
   \rqbraces{\<pr\'eambule>\\cr \<col.>\\cr $\ldots$ \<col.>\\cr}}}
\par}
\explain
Cette commande produit un \minref{alignement} vertical
constitu\'e d'une
suite de colonnes, o\`u chaque colonne \`a son tour contient une suite
d'entr\'ees de rang\'ee.
\TeX\ ajuste les hauteurs des entr\'ees de rang\'ee pour accommoder la plus longue
dans chaque rang\'ee.

Un alignement vertical
n'appara\^\i t que quand \TeX\ est dans un \minref{mode} horizontal.
Puisque les alignements verticaux sont (a)~conceptuellement assez difficiles et
(b)~peu souvent utilis\'es, nous vous recommandons d'ap\-prendre les
alignements en g\'en\'eral
(\xref{alignement}) et la commande |\halign| (voir ci-dessus) avant
d'essayer d'utiliser la commande |\valign|.

Un alignement consiste en un ^{pr\'eambule}
suivi par le texte \`a aligner. Le pr\'eambule,
qui d\'ecrit la forme des colonnes qui le suivent,
consiste en une
suite de patrons de rang\'ee, s\'epar\'es par `|&|' et termin\'e
par |\cr|.
\bix^^{patron}
Chaque colonne consiste en une suite d'entr\'ee 
de ^{rang\'ee}, s\'epar\'ees aussi par `|&|' et termin\'ees par |\cr|. Dans
un patron, `|#|' indique o\`u \TeX\ doit ins\'erer le
texte correspondant \`a l'entr\'ee de rang\'ee. 

\TeX\ compose chaque entr\'ee de rang\'ee en mode vertical 
interne, c'est-\`a-dire, comme le contenu d'une \minref{vbox},
et englobe implicitement l'entr\'ee dans un groupe.
Il donne toujours \`a la vbox une profondeur \`a z\'ero.
Tout texte ou autre mat\'eriel en mode horizontal dans une entr\'ee de rang\'ee met alors \TeX\
en mode horizontal ordinaire.
C'est juste une application des r\`egles g\'en\'erales du fonctionnement de \TeX\
en mode vertical interne.
Les param\`etres de paragraphage habituels s'appliquent dans ce cas~:
l'entr\'ee de rang\'ee a une indentation
initiale de |\parindent| (\xref\parindent) et 
les \minref{ressorts} 
|\leftskip| et |\rightskip| (\xref\leftskip)
s'appliquent \`a ses lignes.

Notez en particulier qu'une entr\'ee de rang\'ee contenant du texte a
une largeur de |\hsize| (\xref\hsize).  A moins que vous mettiez
|\hsize| \`a la largeur de rang\'ee que vous voulez, vous rencontrerez
vraisemblablement des ``overfull \minref{hbox}'' ou trouverez que
la premi\`ere colonne occupe la largeur de la page enti\`ere, ou les deux.
\eix^^{entr\'ee (colonne ou rang\'ee)}

Normalement, vous devrez inclure un \minref{strut}
^^{struts//dans des alignements verticaux}
dans chaque
patron pour que les rang\'ees n'apparaissent pas comme un r\'esultat 
des hauteurs variables des entr\'ees de l'alignement.  Vous
pouvez produire un strut avec la commande |\strut|.

La forme |to| de cette commande demande \`a \TeX\
d'\'etendre l'extension verticale de l'alignement \`a \<dimension>,
en ajustant l'espace entre les rang\'ees.
La forme |spread| de cette commande demande \`a \TeX\
d'agrandir l'alignement de \<dimension> par rapport \`a sa hauteur naturelle.
Ces formes sont comme les formes correspondantes de |\vbox| \ctsref\vbox.

\example
{\hsize=1in \parindent=0pt
\valign{#\strut&#\strut&#\strut&#\strut\cr
   bernaise&curry&hoisin&hollandaise\cr
   ketchup&marinara&mayonnaise&mustard\cr
   rarebit&tartar\cr}}
|
\produces
{\hsize=1in \parindent=0pt \leftskip=0pt
\valign{#\strut&#\strut&#\strut&#\strut\cr
   bernaise&curry&hoisin&hollandaise\cr
   ketchup&marinara&mayonnaise&mustard\cr
   rarebit&tartar\cr}}
\nextexample
% same thing but without struts (shows why you need them)
{\hsize=1in \parindent=0pt
\valign{#&#&#&#\cr
   bernaise&curry&hoisin&hollandaise\cr
   ketchup&marinara&mayonnaise&mustard\cr
   rarebit&tartar\cr}}
|
\produces
{\hsize=1in \parindent=0pt \leftskip=0pt
\valign{#&#&#&#\cr
   bernaise&curry&hoisin&hollandaise\cr
   ketchup&marinara&mayonnaise&mustard\cr
   rarebit&tartar\cr}}
\endexample
\enddesc

\begindesc
\cts ialign {}
\explain
Cette commande r\'eagit comme ^|\halign|,
sauf qu'elle met d'abord le ressort |\tabskip| \`a z\'ero et
met |\everycr| \`a vide.
\enddesc

\begindesc
\cts cr {}
\explain
Cette commande termine le pr\'eambule d'un alignement horizontal ou 
vertical, une rang\'ee d'un alignement horizontal ou tabul\'e,
ou une colonne d'un alignement vertical.
Vous pouvez demander \`a \TeX\ de faire certaines actions \`a chaque fois qu'il rencontre un |\cr|
en chargeant la valeur du param\`etre ^|\everycr| \ctsref\everycr.
\enddesc

\begindesc
\cts endline {}
\explain
Cette commande est un synonyme de la commande ^|\cr|.  Elle est utile quand
vous avez red\'efini |\cr| mais devez encore acc\'eder \`a sa d\'efinition originale.
\enddesc

\begindesc
\cts crcr {}
\explain
Cette commande r\'eagit comme ^|\cr|, sauf que \TeX\ l'ignore si
elle arrive imm\'ediatement apr\`es un |\cr| ou un ^|\noalign|.
Sa principale application est comme mesure de s\'ecurit\'e pour \'eviter un message d'erreur ``misleading''
caus\'e par une \minref{macro} qui attend un argument se terminant par |\cr|.
Si vous mettez |\crcr| apr\`es le `|#|$n$' qui d\'enote un tel argument
dans la d\'efinition de macro, la
macro marchera proprement, que l'argument se termine par |\cr| ou non.
\enddesc

\begindesc
\cts omit {}
\explain
Cette commande demande \`a \TeX\ d'ignorer un patron dans un \minref{alignement} 
horizontal ou vertical pendant le traitement d'une entr\'ee de colonne ou de rang\'ee particuli\`ere
respectivement.
|\omit| doit appara\^\i tre comme le premier \'el\'ement dans une entr\'ee de colonne ou de rang\'ee~; en effet,
il remplace le patron du pr\'eambule avec le simple
patron `|#|'.


\example
\tabskip = 2em\halign{%
   \hfil\it#\hfil&\hfil#\hfil&#&\hfil\$#\cr
   United States&Washington&dollar&1.00\cr
   \omit \dotfill France\dotfill&Paris&franc&0.174\cr
   Israel&Jerusalem&shekel&0.507\cr
   Japan&Tokyo&yen&0.0829\cr}
|
\produces
\tabskip = 2em\halign{%
   \hfil\it#\hfil&\hfil#\hfil&#&\hfil\$#\cr
   United States&Washington&dollar&1.00\cr
   \omit \dotfill France\dotfill&Paris&franc&0.174\cr
   Israel&Jerusalem&shekel&0.507\cr
   Japan&Tokyo&yen&0.0829\cr}
\nextexample
{\hsize=1.2in \parindent=0pt 
\valign{(#)\strut&(#)\strut&(#)\strut&(#)\strut\cr
   bernaise&curry&hoisin&hollandaise\cr
   ketchup&\omit\strut{\bf MARINARA!!}&mayonnaise&mustard\cr

   rarebit&tartar\cr}}
|
\produces
{\hsize=1.2in \parindent=0pt \leftskip=0pt
\valign{(#)\strut&(#)\strut&(#)\strut&(#)\strut\cr
   bernaise&curry&hoisin&hollandaise\cr
   ketchup&\omit\strut{\bf MARINARA!}&mayonnaise&mustard\cr
   rarebit&tartar\cr}}
\endexample
\enddesc

\begindesc
\cts span {}
\explain
La signification de cette commande d\'epend suivant qu'elle appara\^\i t soit dans un pr\'eambule
soit dans une entr\'ee d'alignement.
\ulist
\li Normalement, \TeX\ ne d\'eveloppe pas de tokens dans le pr\'eambule
quand il le lit.
Mettre |\span| au d\'ebut d'un token dans le pr\'eambule rend
ce token d\'eveloppable imm\'ediatement selon les r\`egles usuelles de d\'eveloppement
de \minref{macro} de \TeX.
\li Mettre |\span| \`a la place de `|&|' entre deux entr\'ees de colonne ou de rang\'ee
rend ces colonnes ou rang\'ees combin\'ees.
Pour un alignement horizontal,
la largeur de la colonne combin\'ee est la somme des
largeurs des colonnes incluses.
Pour un alignement vertical,
la hauteur de la rang\'ee combin\'ee est la somme des
hauteurs des rang\'ees incluses.
Le patron des colonnes ou rang\'ees combin\'ees forme un seul groupe,
donc des commandes de chargement de police pr\'ec\'edant un |\span| affectent tout jusqu'au 
prochain `|&|'.
\endulist
\noindent
|\span| est rarement utile d'elle-m\^eme en dehors d'un patron,
mais procure le m\'ecanisme de base 
pour d\'efinir ^|\multispan|.
\enddesc

\begindesc
\cts multispan {\<nombre>}
\explain
Cette commande demande \`a \TeX\ que les \<nombre> 
colonnes suivantes d'une rang\'ee d'un alignement horizontal,
ou
que les \<nombre> rang\'ees dans une colonne d'un alignement vertical,
doivent \^etre combin\'ees en une seule colonne ou rang\'ee (comme avec
|\span|) et que leur patron doit \^etre omis (comme avec |\omit|).

\example
\tabskip = 13pt\halign{%
   \hfil\it#\hfil&\hfil#\hfil&#&\hfil\$#\cr
   United States&Washington&dollar&1.00\cr
   France&Paris&franc&0.174\cr
   Israel&Jerusalem &
      \multispan 2 \hfil\it(no information)\hfil \cr
   Japan&Tokyo&yen&0.0829\cr}
|
\produces
\tabskip = 13pt\halign{%
   \hfil\it#\hfil&\hfil#\hfil&#&\hfil\$#\cr
   United States&Washington&dollar&1.00\cr
   France&Paris&franc&0.174\cr
   Israel&Jerusalem &
      \multispan 2 \hfil\it(no information)\hfil \cr
   Japan&Tokyo&yen&0.0829\cr}
\nextexample
{\hsize=1.2in \parindent=0pt
\valign{(#)\strut&(#)\strut&(#)\strut&(#)\strut\cr
   bernaise&curry&hoisin&hollandaise\cr
   \multispan 3$$\left\{{{\rm ketchup}\atop{\rm marinara}}
      \right\}$$&mustard\cr
   rarebit&tartar\cr}}
|
\produces
{\hsize=1.2in \parindent=0pt \leftskip=0pt
\valign{(#)\strut&(#)\strut&(#)\strut&(#)\strut\cr
   bernaise&curry&hoisin&hollandaise\cr
   \multispan 3$$\left\{{{\rm ketchup}\atop{\rm marinara}}
      \right\}$$&mustard\cr
   rarebit&tartar\cr}}
\endexample

\eix^^{patron}
\enddesc

\begindesc
\cts noalign {\rqbraces{\<mat\'eriel en mode vertical>}}
\aux\cts noalign {\rqbraces{\<mat\'eriel en mode horizontal>}}
\explain
Cette commande ins\`ere
\<mat\'eriel en mode vertical>
apr\`es la rang\'ee cou\-rante d'un \minref{alignement} horizontal ou
\<mat\'eriel en mode horizontal> apr\`es la colonne courante d'un \minref{alignement} 
vertical.
Le mat\'eriel peut \^etre du texte, un ressort, un filet ou tout autre chose.

L'utilisation la plus commune de |\noalign|
est de mettre de l'espace suppl\'ementaire apr\`es une rang\'ee ou une colonne.
Si vous voulez mettre de l'espace suppl\'ementaire apr\`es \emph{toutes\/} les rang\'ees d'un alignement 
horizontal, utilisez ^|\openup| (\xref\openup).

\example
\halign{%
   \hfil\it#\hfil\tabskip=2em&\hfil#\hfil&#&
      \hfil\$#\tabskip=0em\cr
   % The \tabskip changes prevent the rule below
   % from sticking out.
   United States&Washington&dollar&1.00\cr
   France&Paris&franc&0.174\cr
   \noalign{\smallskip\hrule\smallskip}
   Israel&Jerusalem&shekel&0.507\cr
   Japan&Tokyo&yen&0.0829\cr}
|
\produces
\halign{%
   \hfil\it#\hfil\tabskip=2em&\hfil#\hfil&#&
      \hfil\$#\tabskip=0em\cr
   % The \tabskip changes prevent the rule below
   % from sticking out.
   United States&Washington&dollar&1.00\cr
   France&Paris&franc&0.174\cr
   \noalign{\smallskip\hrule\smallskip}
   Israel&Jerusalem&shekel&0.507\cr
   Japan&Tokyo&yen&0.0829\cr}
\nextexample
{\hsize=1in \parindent=0pt
\valign{#\strut&#\strut&#\strut&#\strut\cr
   \noalign{\vrule width 2pt\quad}
   bernaise&curry&hoisin&hollandaise\cr
   \noalign{\vrule width 2pt\quad}
   ketchup&marinara&mayonnaise&mustard\cr
   \noalign{\vrule width 2pt\quad}
   rarebit&tartar\cr
   \noalign{\vrule width 2pt\quad}}}
|
\produces
\medskip
{\hsize=1in \parindent=0pt
\valign{#\strut&#\strut&#\strut&#\strut\cr
   \noalign{\vrule width 2pt\quad}
   bernaise&curry&hoisin&hollandaise\cr
   \noalign{\vrule width 2pt\quad}
   ketchup&marinara&mayonnaise&mustard\cr
   \noalign{\vrule width 2pt\quad}
   rarebit&tartar\cr
   \noalign{\vrule width 2pt\quad}}}
\endexample
\enddesc

\begindesc
\cts tabskip {\param{ressort}}
\explain
Ce param\`etre sp\'ecifie le montant de ressort horizontal ou vertical  
que \TeX\ met entre les 
colonnes d'un alignement horizontal ou entre les
rang\'ees d'un alignement vertical.
\TeX\ met aussi le ressort |\tabskip|
\`a gauche de la premi\`ere colonne et \`a droite de la derni\`ere colonne
d'un alignement horizontal,
avant la premi\`ere rang\'ee 
et apr\`es la derni\`ere rang\'ee d'un alignement vertical.
Vous pouvez changer |\tabskip| dans un patron---%
le changement affectera le ressort associ\'e avec tous les
|&| suivant aussi bien que le ressort apr\`es la derni\`ere rang\'ee ou colonne.
\example
\halign to 3.5in{%
   \hfil\it#\tabskip = 2em plus 8pt
      \hfil&\hfil#\hfil&#\tabskip = 1em
      &\hfil\$#\tabskip = 0em\cr
   United States&Washington&dollar&1.00\cr
   France&Paris&franc&0.174\cr
   Israel&Jerusalem&shekel&0.507\cr
   Japan&Tokyo&yen&0.0829\cr}
|
\produces
\halign to 3.5in{%
   \hfil\it#\tabskip = 2em plus 8pt
      \hfil&\hfil#\hfil&#\tabskip = 1em
      &\hfil\$#\tabskip = 0em\cr
   United States&Washington&dollar&1.00\cr
   France&Paris&franc&0.174\cr
   Israel&Jerusalem&shekel&0.507\cr
   Japan&Tokyo&yen&0.0829\cr}
\nextexample
{\hsize = 1in \parindent=0pt \tabskip=5pt
\valign{#\strut&#\strut\tabskip = 3pt
   &#\strut&#\strut\cr
   bernaise&curry&hoisin&hollandaise\cr
   ketchup&marinara&mayonnaise&mustard\cr
   rarebit&tartar\cr}}
|
\produces
{\hsize = 1in \parindent=0pt \tabskip=5pt
\valign{#\strut&#\strut\tabskip = 3pt
   &#\strut&#\strut\cr
   bernaise&curry&hoisin&hollandaise\cr
   ketchup&marinara&mayonnaise&mustard\cr
   rarebit&tartar\cr}}
\endexample
\enddesc

\begindesc
\cts hidewidth {}
\explain
Cette commande demande \`a \TeX\ d'ignorer la largeur de l'entr\'ee de la colonne suivante dans un
alignement horizontal.  C'est pratique quand vous avez une entr\'ee qui est plus longue
que la plupart des autres dans la m\^eme colonne
et que vous pr\'ef\'erez avoir cette entr\'ee sortant de la colonne plut\^ot
que rendre toutes les entr\'ees
de la colonne plus larges.  Si le |\hidewidth| est \`a gauche de
l'entr\'ee, l'entr\'ee d\'ebordera \`a gauche~; si le |\hidewidth| est \`a
droite de l'entr\'ee, l'entr\'ee d\'ebordera \`a droite.
\example
\tabskip = 25pt\halign{%
   \hfil\it#\hfil&\hfil#\hfil&#&\hfil\$#\cr
   United States&\hidewidth Washington&
      dollar&1.00\cr
   France&Paris&franc&0.174\cr
   Israel&Jerusalem&shekel&0.507\cr
   Japan&Tokyo&yen&0.0829\cr}
|
\produces
\tabskip = 25pt\halign{%
   \hfil\it#\hfil&\hfil#\hfil&#&\hfil\$#\cr
   United States&\hidewidth Washington&
      dollar&1.00\cr
   France&Paris&franc&0.174\cr
   Israel&Jerusalem&shekel&0.507\cr
   Japan&Tokyo&yen&0.0829\cr}
\endexample
\enddesc

\begindesc
\cts everycr {\param{liste de token}}
\explain
\TeX\ d\'eveloppe \<token list> \`a chaque fois qu'il ex\'ecute
un |\cr|---\`a la fin de chaque pr\'eambule,
\`a la fin de chaque rang\'ee d'un alignement horizontal
et \`a la fin de chaque colonne d'un alignement vertical.
La commande |\everycr| est d\'evelopp\'ee juste apr\`es le |\cr|.
Ainsi vous pouvez demander \`a \TeX\ 
d'ex\'ecuter certaines commandes \`a la fin d'un pr\'eambule, d'une rang\'ee ou d'une colonne en
assignant une liste de commandes \`a |\everycr|.

Les tokens |\everycr| ne doivent pas inclure d'autres commandes que |\no!-align|.
Ceci parce que 
les tokens |\everycr| r\'eappara\^\i tront apr\`es le dernier
|\cr| de l'alignement. Une commande autre qu'un
|\noalign| fera penser \`a \TeX\ qu'il commence une nouvelle
rang\'ee ou colonne.
\TeX\ se plaindra d'un |\cr| manquant, ins\`erera
un |\cr|, ins\`erera les tokens |\everycr| \`a nouveau et r\'ep\`etera ces actions
ind\'efiniment. 
\example

\everycr={\noalign{\smallskip\hrule\smallskip}}
\halign{#\tabskip = 11pt&\hfil#\hfil&\hfil#\hfil
      \tabskip = 0pt\cr
   $1$&one&first\cr
   $2$&two&second\cr
   $3$&three&third\cr}
|
\produces
\medskip
\everycr={\noalign{\smallskip\hrule\smallskip}}
\halign{#\tabskip = 11pt&\hfil#\hfil&\hfil#\hfil
      \tabskip = 0pt\cr
   $1$&one&first\cr
   $2$&two&second\cr
   $3$&three&third\cr}
\endexample

\eix^^{alignements//commandes pour}
\enddesc

\enddescriptions


\endchapter
\byebye
