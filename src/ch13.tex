\chapter{Theorem of Mean Value; Indeterminant Forms}
\label{ch:13}

\section{Rolle's Theorem}
\label{sec:13:01}

\section{The Mean Value Theorem}
\label{sec:13:02}

\section{The Extended Mean Value Theorem}
\label{sec:13:03}

\section{Exercises}
\label{sec:13:04}

\section{Maxima and Minima Treated Analytically}
\label{sec:13:05}

\section{Exercises}
\label{sec:13:06}

\section{Indeterminate Forms}
\label{sec:13:07}

\section{Evaluation of a Function Taking on an Indeterminate Form}
\label{sec:13:08}

\section{Evaluation of the Indeterminate Form $\frac{0}{0}$}
\label{sec:13:09}

\subsection{Rule for Evaluating the Indeterminate Form $\frac{0}{0}$}
\label{subsec:13:09:01}

\subsection{Exercises}
\label{subsec:13:09:02}

\section{Evaluation of the Indeterminate Form $\frac{\infty}{\infty}$}
\label{sec:13:10}

\section{Evaluation of the Indeterminate Form $0 \cdot \infty$}
\label{sec:13:11}

\section{Evaluation of the Indeterminate Form $\infty - \infty$}
\label{sec:13:12}

\subsection{Exercises}
\label{sec:13:12:01}

\section{Evaluation of the Indeterminate Forms $0^0$, $1^\infty$, $\infty^0$}
\label{sec:13:13}

\section{Exercises}
\label{sec:13:14}

\section{Application: Using Taylor's Theorem to Approximate Functions}
\label{sec:13:15}

\section{Example/Application: Finite Difference Schemes}
\label{sec:13:16}
