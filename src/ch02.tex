\chapter{Variables and Functions}
\label{ch:02}

\section{Variables and Constants}
\label{sec:02:01}

\index{variable}
\index{constant}
A \emph{variable} is a quantity to which an unlimited number of values can be
assigned. Variables are denoted by the later letters of the alphabet. Thus, in
the equation of a straight line,
\[
\frac{x}{a} + \frac{y}{b} = 1
\]
$x$ and $y$ may be considered as the variable co\"ordinates of a point moving
along the line. A quantity whose value remains unchanged is called a
\emph{constant}.

Numerical or absolute constants retain the same values in all problems, as $2$,
$5$, $\sqrt{7}$, $\pi$, etc.

\index{parameters}
\emph{Arbitrary constants}, or \emph{parameters}, are constants to which any
one of an unlimited set of numerical values may be assigned, and they are
supposed to have these assigned values throughout the investigation. They are
usually denoted by the earlier letters of the alphabet. Thus, for every pair of
values arbitrarily assigned to $a$ and $b$, the equation
\[
\frac{x}{a} + \frac{y}{b} = 1
\]
represents some particular straight line.

\section{Interval of a Variable}
\label{sec:02:02}

Very often we confine ourselves to a portion only of the number system. For
example, we may restrict our variable so that it shall take on only such values
as lie between $a$ and $b$, where $a$ and $b$ may be included, or either or
both excluded. We shall employ the symbol $\left \lbrack a,\ b \right \rbrack$,
$a$ being less than $b$, to represent the numbers $a,\ b$, and all the numbers
between them, unless otherwise stated. This symbol
$\left \lbrack a,\ b \right \rbrack$ is read the interval from $a$ to $b$.

\section{Continuous Variation}
\label{sec:02:03}

A variable $x$ is said to vary continuously through an interval $\left \lbrack
a, b \right \rbrack$, when $x$ starts with the value $a$ and increases until it
takes on the value $b$ in such a manner as to assume the value of every number
between $a$ and $b$ in the order of their magnitudes. This may be illustrated
geometrically as follows:

\begin{sagesilent}
f = lambda x:0
G1 = plot(f, -5, 6, thickness=3, rgbcolor=(0.0,1,0.5), axes=False)
G2 = plot(f, 0, 5, thickness=5, rgbcolor=(0.5,0.3,0.5), axes=False)

t1 = text('O', (-4.5, -0.25))
t2 = text('A', (0.0, -0.25))
t3 = text('a', (0.0, 0.25))
t4 = text('B', (5.0, -0.25))
t5 = text('b', (5.0, 0.25))
t6 = text('P', (2.0, -0.25))
t7 = text('x', (2.0, 0.25))

p1 = disk((0.0, 0.0), 0.10, (0, 2*pi))
p2 = disk((5.0, 0.0), 0.10, (0, 2*pi))
p3 = disk((-4.5, 0.0), 0.10, (0, 2*pi))
p4 = disk((2.0, 0.0), 0.10, (0, 2*pi))

p = G1+G2+t1+t2+t3+t4+t5+t6+t7+p1+p2+p3+p4
\end{sagesilent}

\begin{figure}[h]
\begin{center}
\sageplot[width=10cm]{p}
\end{center}
\caption{Interval from $A$ to $B$}
\label{fig:02:01}
\end{figure}

The origin being at $O$, layoff on the straight line the points 
$A$ and $B$ corresponding to the numbers $a$ and $b$. Also 
let the point $P$ correspond to a particular value of the variable $x$. 
Evidently the interval $\left \lbrack a, b \right \rbrack$ is 
represented by the segment $AB$. Now as $x$ varies continuously 
from $a$ to $b$ inclusive, i.e. through the interval 
$\left \lbrack a, b \right \rbrack$, the point $P$ 
generates the segment $AB$.

\section{Functions}
\label{sec:02:04}

\index{function}
When two variables are so related that the value of the first variable
depends on the value of the second variable, then the first variable is
said to be a \emph{function} of the second variable.

Nearly all scientific problems deal with quantities and relations of this sort,
and in the experiences of everyday life we are continually meeting conditions
illustrating the dependence of one quantity on another. For instance, the
weight a man is able to lift depends on his strength, other things being equal.
Similarly, the distance a boy can run may be considered as depending on the
time. Or, we may say that the area of a square is a function of the length of a
side, and the volume of a sphere is a function of its diameter.

\begin{Verbatim}[fontsize=\tiny,fontfamily=courier,fontshape=tt,frame=single,label=\SAGE]
sage: I1 = interval(1,3)
sage: I2 = interval(2,6)
sage: I3 = interval(min(I2),max(I1)) # the intersection
sage: P1 = plot(0, xmin=min(I1), xmax = max(I1), thickness=10, rgbcolor=(1,0,0),linestyle="--")
sage: P2 = plot(0, xmin=min(I2), xmax = max(I2), thickness=10, rgbcolor=(0,1,0),linestyle=":")
sage: P3 = plot(0, xmin=min(I3), xmax = max(I3), thickness=10, rgbcolor=(1,1,0))
sage: show(P1+P2+P3)
\end{Verbatim}

\begin{sagesilent}
I1 = interval(1,3)
I2 = interval(2,6)
I3 = interval(min(I2),max(I1)) # the intersection
P1 = plot(0, xmin=min(I1), xmax = max(I1), thickness=10, rgbcolor=(1,0,0),linestyle="--")
P2 = plot(0, xmin=min(I2), xmax = max(I2), thickness=10, rgbcolor=(0,1,0),linestyle=":")
P3 = plot(0, xmin=min(I3), xmax = max(I3), thickness=10, rgbcolor=(1,1,0))
p = P1+P2+P3
\end{sagesilent}

\begin{figure}[h]
\begin{center}
\sageplot[width=10cm]{p}
\end{center}
\caption{Interval $[1,3]$ is dashed, $[2,5]$ is dotted, $[1,3]\cal [2,5]=[2,3]$ is solid.}
\label{fig:02:02}
\end{figure}

\section{Independent and Dependent Variables}
\label{sec:02:05}

\index{argument}
\index{independent variable}
\index{dependent variable}
\index{function}
The second variable, to which values may be assigned at pleasure within limits
depending on the particular problem, is called the \emph{independent variable},
or \emph{argument}; and the first variable, whose value is determined as soon
as the value of the independent variable is fixed, is called the
\emph{dependent variable}, or \emph{function}.

Frequently, when we are considering two related variables, it is in our power to
fix upon whichever we please as the independent variable; but having once made
the choice, no change of independent variable is allowed without certain
precautions and transformations.

One quantity (the dependent variable) may be a function of two or more other
quantities (the independent variables, or arguments).  For example, the cost of
cloth is a function of both the quality and quantity; the area of a triangle is
a function of the base and altitude; the volume of a rectangular parallelepiped
is a function of its three dimensions.

In the \SAGE example below, $t$ is the independent variable and $f$ is the
dependent variable. Though this example uses differentiation (\SAGE's {\tt
f.diff} notation), you don't need to understand how to differentiate to
appreciate the syntax \SAGE uses to express the command.

\begin{Verbatim}[fontsize=\scriptsize,fontfamily=courier,fontshape=tt,frame=single,label=\SAGE]
sage: t = var('t')
sage: f = function('f', t)
sage: f(t).diff(t)
diff(f(t), t, 1)
sage: (2*f(t)+3).diff(t)
2*diff(f(t), t, 1)
sage: (f(t)^2).diff(t)
2*f(t)*diff(f(t), t, 1)
sage: f = cos
sage: f(pi/2)
0
sage: f(t).diff(t)
-sin(t)
\end{Verbatim}

\section{Notation of Functions}
\label{sec:02:06}

The symbol $f \left( x \right)$ is used to denote a function of $x$, and is
read ``$f$ of $x$''. In order to distinguish between different functions, the
prefixed letter is changed, as $F \left( x \right)$, $\phi \left( x \right)$,
$f' \left( x \right)$, etc.

During any investigation the same functional symbol always indicates the same
law of dependence of the function upon the variable.  In the simpler cases this
law takes the form of a series of analytical operations upon that variable.
Hence, in such a case, the same functional symbol will indicate the same
operations or series of operations, even though applied to different quantities.
Thus, if
\[
  f \left( x \right) = x^2 - 9x + 14,
\]
then 	
\[
  f \left( y \right) 	= y^2 - 9y + 14.
\]
Also 	
\[
  f \left( a \right) 	= a^2 - 9a + 14,
\]
\[
  f \left( b + 1 \right) =  \left( b + 1 \right)^2 - 9 \left( b + 1 \right) + 14
  = b^2 - 7b + 6,
\]
\[
  f \left( 0 \right) 	= 0^2 - 9 \cdot 0 + 14 = 14,
\]
\[
  f \left( -1 \right) 	=  \left( -1 \right)^2 - 9 \left( -1 \right) + 14 = 24,
\]
\[
  f \left( 3 \right) 	= 3^2 - 9 \cdot 3 + 14 = - 4,
\]
\[
  f \left( 7 \right) 	= 7^2 - 9 \cdot 7 + 14 = 0,
\]
etc. Similarly, $\phi \left( x,\ y \right)$ denotes a function of $x$ and $y$,
and is read ``$\phi$ of $x$ and $y$''. If 	
\[
  \phi \left( x,\ y \right) 	= \sin \left( x + y \right),
\]
then 	
\[
  \phi \left( a,\ b \right) 	= \sin \left( a + b \right),
\]
and 	
\[
  \phi \left( \frac{\pi}{2}, 0 \right)	= \sin \frac{\pi}{2} = 1.
\]
Again, if 	
\[
  F \left(  x,\ y,\ z  \right) 	= 2x + 3y - 12z,
\]
then 	
\[
  F \left( m,\ -m,\ m \right) 	= 2m - 3m - 12m = - 13m.
\]
and 	
\[
  F \left( 3,\ 2,\ 1 \right) 	= 2 \cdot 3 + 3 \cdot 2 - 12 \cdot 1 = 0.
\]
Evidently this system of notation may be extended indefinitely.

You can define a function in \SAGE in several ways:

\begin{Verbatim}[fontsize=\scriptsize,fontfamily=courier,fontshape=tt,frame=single,label=\SAGE]
sage: x,y = var("x,y")
sage: f = log(sqrt(x))
sage: f(4)
log(4)/2
sage: f(4).simplify_log()
log(2)
sage: f = lambda x: (x^2+1)/2
sage: f(x)
(x^2 + 1)/2
sage: f(1)
1
sage: f = lambda x,y: x^2+y^2
sage: f(3,4)
25
sage: R.<x> = PolynomialRing(CC,"x")
sage: f = x^2+2
sage: f.roots()
[(1.41421356237309*I, 1), (2.77555756156289e-17 - 1.41421356237309*I, 1)]
\end{Verbatim}

\section[Values for which a Function is Defined]{Values of the Independent
Variable for which a Function is Defined}
\label{sec:02:07}

Consider the functions
\[
  x^2 - 2x + 5,\qquad \sin x,\qquad \arctan x
\]
of the independent variable $x$. Denoting the dependent variable in each case
by $y$, we may write
\[
  y = x^2 - 2 x + 5, y = \sin x,\ y = \arctan x.
\]
In each case $y$ (the value of the function) is known, or, as we say, defined,
for all values of $x$. This is not by any means true of all functions, as the
following examples illustrating the more common exceptions will show.

\begin{equation}
  y = \frac{a}{x - b}
  \label{eqn:02:01}
\end{equation}
Here the value of $y$ (i.e. the function) is defined for all values of $x$
except $x = b$. When $x = b$ the divisor becomes zero and the value of $y$
cannot be computed from (\ref{eqn:02:01}).  Any value might be assigned to
the function for this value of the argument.

\begin{equation}
  y = \sqrt{x}.
  \label{eqn:02:02}
\end{equation}
In this case the function is defined only for positive values of $x$.
Negative values of $x$ give imaginary values for $y$, and these must be
excluded here, where we are confining ourselves to real numbers only.

\begin{equation}
  y = \log_a{x}. \qquad a > 0
  \label{eqn:02:03}
\end{equation}
Here $y$ is defined only for positive values of $x$. For negative values of $x$
this function does not exist (see \ref{sec:XXX}).

\begin{equation}
  y = \arcsin x,\ y = \arccos x.
  \label{eqn:02:04}
\end{equation}
Since sines, and cosines cannot become greater than $+1$ nor less than $-1$, it
follows that the above functions are defined for all values of $x$ ranging from
$-1$ to $+1$ inclusive, but for no other values.

\begin{Verbatim}[fontsize=\scriptsize,fontfamily=courier,fontshape=tt,frame=single,label=\SAGE]
sage: t = var("t'')
sage: f = function('f', t)
sage: g = function('g', t)
sage: f = sin
sage: g = asin
sage: f(g(t))
t
sage: g(f(t))
t
sage: g(f(0.2))
0.200000000000000
\end{Verbatim}

\section{Exercises}
\label{sec:02:08}

\begin{enumerate}
\item
Given $f \left( x \right) = x^3 - 10x^2 + 31x - 30$; show that

\[
  f \left( 0 \right) 	= -30,\ \ \ \ f \left( y \right) = y^3 - 10y^2 + 31y - 30,
\]
\[
  f \left( 2 \right) = 0,\ \ \ \  f \left( a \right) 	= a^3 - 10a^2 + 31a - 30,
\]
\[
  f \left( 3 \right) =	f \left( 5 \right),\ \ \ \ 
  f \left( yz \right) = y^3z^3 - 10y^2z^2 + 31yz - 30,
\]
\[
  f \left( 1 \right) > f \left(  -3 \right),\ \ \ \ 
  f \left( x -2 \right) = x^3 - 16x^2 + 83x - 140,
\]
\[
  f \left( -1 \right) = 6f \left( 6 \right). 
\]

\item
If $f \left( x \right) = x^3 - 3x + 2$, find
$f \left( 0 \right)$,
$f \left( 1 \right)$,
$f \left( -1 \right)$,
$f \left ( -\frac{1}{2} \right )$,
$f \left ( \frac{4}{3} \right )$.

\item
If $f \left( x \right) = x^3 - 10x^2 + 31x - 30$, and
$\phi  \left( x \right) = x^4 -55x^2 -210x -216$, show that
$f \left( 2 \right) = \phi \left(  - 2 \right)$,
$f \left( 3 \right) = \phi \left(  - 3 \right),f \left( 5 \right) = \phi \left(  - 4 \right)$,
$f \left( 0 \right) + \phi \left( 0 \right) + 246 = 0$.

\item
If $F \left( x \right) = 2x$, find $F \left( 0 \right)$,\ $F \left( -3 \right)$,\ 
$F \left ( \frac{1}{3} \right )$, $F \left( -1 \right)$.

\item
Given $F(x) 
= x \left( x - 1 \right) \left( x + 6 \right) \left( x - \frac{1}{2} \right) 
\left(x + \frac{5}{4} \right)$, show that
$F \left( 0 \right) = F \left( 1 \right) = F \left( -6 \right) = F \left(\frac{1}{2} \right) 
= F \left( -\frac{5}{4} \right) = 0$.  

\item
If $f \left( m_1 \right) = \frac{m_1 - 1}{m_1 + 1}$, show that
$\frac{f \left( m_1 \right) - f \left( m_2 \right)}{1 + f \left( m_1 \right)f \left( m_2 \right)} = \frac{m_1 - m_2}{1 + m_1 m_2}$.

\item
If $\phi \left( x \right) = a^x$, show that $\phi \left( y \right) \cdot \phi \left( z \right) = \phi \left( y + z \right)$.

\item
Given $\phi \left( x \right) = \log \frac{1 - x}{1 + x}$, show that
$ \phi \left( x \right) + \phi \left( y \right) = \phi \left ( \frac{x + y}{1 + xy} \right )$. 

\item
If $f \left( \phi  \right) = \cos\phi$, show that
$ f \left( \phi  \right) = f \left(  - \phi  \right) = - f \left( \pi- \phi \right) = - f \left( \pi + \phi \right)$.

\item
 If $F \left( \theta \right) = \tan\theta$, show that
$F \left( 2\theta \right) = \frac{2F \left( \theta \right)}{1 - [ F \left( \theta \right) ]^2}$. 

\item
 Given $\psi \left( x \right) = x^{2n} + x^{2m} + 1$, show that
$ \psi \left( 1 \right) = 3$,\ $\psi \left( 0 \right) = 1$, and $\psi \left( a \right) = \psi(-a)$. 

\item
If $f \left( x \right) = \frac{2x - 3}{x + 7}$, find $f \left( \sqrt{2} \right)$.

\end{enumerate}

Here's how to verify the double angle identity for $\tan$ in
Exercise 10 above:

\begin{Verbatim}[fontsize=\scriptsize,fontfamily=courier,fontshape=tt,frame=single,label=\SAGE]
sage: theta = var("theta")
sage: tan(2*theta).expand_trig()
2*tan(theta)/(1 - tan(theta)^2)
\end{Verbatim}
