\chapter{Collection of Formulas}
\label{ch:01-collection-of-formulas}

% ========================================
\section{Formulas for Reference}
\label{sec:1:1:formulas-for-reference}
For the convenience of the student we give the following list of elementary
formulas from Algebra, Geometry, Trigonometry, and Analytic Geometry.
\begin{enumerate}

% #1 ==========
\item
Binomial Theorem ($n$ being a postive integer):
\[
\begin{aligned}
\left(a + b\right)^n =
a^n + na^{n - 1}b &
+ \frac{n\left(n - 1\right)}{2!} a^{n - 2}b^2
+ \frac{n\left(n - 1\right)\left(n - 2\right)}{3!}a^{n - 3}b^3+\cdots\\ &
+ \frac{n\left(n - 1\right)\left(n - 2\right)\cdots
    \left(n - r + 2\right)}{\left(r - 1\right)!}a^{n - r + 1}b^{r - 1}
+\cdots
\end{aligned}
\]

% #2 ====================
\item
$n! = 1 \cdot 2 \cdot 3 \cdot 4 \cdots \left(n - 1\right)n$.

% #3 ====================
\item
In the quadratic equation $ax^2 + bx + c = 0$,
\[
\begin{aligned}
\textnormal{when }b^2 - 4ac & > 0\textnormal{, the roots are real and unequal;}\\
\textnormal{when }b^2 - 4ac & = 0\textnormal{, the roots are real and equal;}\\
\textnormal{when }b^2 - 4ac & < 0\textnormal{, the roots are imaginary.}
\end{aligned}
\]

% #4 ====================
\item
When a quadratic equation is reduced to the form $x^2 + px + q = 0$,
\[
\begin{aligned}
p= & \textnormal{sum of roots with sign changed, and}\\
q= & \textnormal{product of roots.}
\end{aligned}
\]

% #5 ====================
\item
In an arithmetical series, $a$, $a + d$, $a + 2d$, \ldots,
\[
s = \sum_{i = 0}^{n - 1} a + id
  = \frac{n}{2} \left[2a + \left(n - 1\right) d \right]
\textnormal{.}
\]

% #6 ====================
\item
In a geometrical series, $a$, $ar$, $ar^2$, \ldots,
\[
s = \sum_{i = 0}^{n - 1}ar^i
  = \frac{a \left(r^n - 1 \right)}{r-1}
\textnormal{.}
\]

% #7 ====================
\item
$\log ab = \log a + \log b$.

% #8 ====================
\item
$\log \frac{a}{b} = \log a - \log b$.

% #9 ====================
\item
$\log a^n = n\log a$.

% #10 ====================
\item
$\log \sqrt[n]{a} = \frac{1}{n} \log a$.

% #11 ====================
\item
$\log 1 = 0$.

% #12 ====================
\item
$\log e = 1$.

% #13 ====================
\item
$\log \frac{1}{a} = -\log a$.

% #14 ====================
\item \footnote{In formulas 14-25, $r$ denotes radius, 
$a$ altitude, $B$ area of base, and $s$ slant height.}
Circumference of circle = $2 \pi r$. 

% #15 ====================
\item
Area of circle = $\pi r^2$.

% #16 ====================
\item
Volume of prism = $Ba$.
 
% #17 ====================
\item
Volume of pyramid = $\frac{1}{3} Ba$.

% #18 ====================
\item
Volume of right circular cylinder = $\pi r^2a$.
 
% #19 ====================
\item
Lateral surface of right circular cylinder = $2 \pi ra$.

% #20 ====================
\item
Total surface of right circular cylinder = $2 \pi r\left(r + a\right)$.

% #21 ====================
\item
Volume of right circular cone = $2 \pi r\left(r + a\right)$.

% #22 ====================
\item
Lateral surface of right circular cone = $\pi rs$.

% #23 ====================
\item
Total surface of right circular cone = $\pi r\left(r + s\right)$.

% #24 ====================
\item
Volume of sphere = $\frac{4}{3}\pi r^3$.

% #25 ====================
 \item
Surface of sphere = $4\pi r^2$.

% #26 ====================
\item
$\sin x = \displaystyle{\frac{1}{\csc x}}$,
$\cos x = \displaystyle{\frac{1}{\sec x}}$,
$\tan x = \displaystyle{\frac{1}{\cot x}}$.

% #27 ====================
\item
$\tan x = \displaystyle{\frac{\sin x}{\cos x}}$,
$\cot x = \displaystyle{\frac{\cos x}{\sin x}}$,

% #28 ====================
\item
$\sin^2 x + \cos^2 x = 1$;
$1 + \tan^2 x = \sec^2 x$;
$1 + \cot^2 x = \csc^2 x$.

% #29 ====================
\item
$\sin x = \cos\left(\frac{\pi}{2} - x\right)$;
$\cos x = \sin\left(\frac{\pi}{2} - x\right)$;
$\tan x = \cot\left(\frac{\pi}{2} - x\right)$.

% #30 ====================
\item
$\sin\left(\pi - x\right) = \sin x$;
$\cos\left(\pi - x\right) = -\cos x$;
$\tan\left(\pi - x\right) = -\tan x$.

% #31 ====================
\item
$\sin\left(x + y\right) = \sin x \cos y + \cos x \sin y$.

% #32 ====================
\item
$\sin\left(x - y\right) = \sin x \cos y - \cos x \sin y$.

% #33 ====================
\item
$\cos\left(x \pm y\right) = \cos x \cos y + \mp \sin x \sin y$.

% #34 ====================
\item
$\tan\left(x + y\right) =
\displaystyle{ \frac{\tan x + \tan y}{1 - \tan x \tan y} }$.

% #35 ====================
\item
$\tan\left(x - y\right) =
\displaystyle{ \frac{\tan x \tan y}{1 + \tan x \tan y} }$.

% #36 ====================
\item
$\sin 2x = 2 \sin x \cos x$;
$\cos 2x = \cos^2x - \sin^2x$;
$\tan 2x = \displaystyle{ \frac{2 \tan x}{1 - \tan^2x} }$.

% #37 ====================
\item
$\sin x = 2 \sin \frac{x}{2} \cos \frac{x}{2}$;
$\cos x = \cos^2 \frac{x}{2} - \sin^2 \frac{x}{2}$;
$\tan x = \displaystyle{ \frac{2 \tan \frac{1}{2} x}{1 - \tan^2 \frac{1}{2}x} }$.

% #38 ====================
\item
$\cos^2 x = \frac{1}{2} + \frac{1}{2} \cos 2x$;
$\sin^2 x = \frac{1}{2} - \frac{1}{2} \cos 2x$.

% #39 ====================
\item
$1 + \cos x = 2 \cos^2 \frac{x}{2}$;
$1 - \cos x = 2 \sin^2 \frac{x}{2}$.

% #40 ====================
\item
$\sin \frac{x}{2} = \pm \displaystyle{ \sqrt{ \frac{1 - \cos x}{2} } }$;
$\cos \frac{x}{2} = \pm \displaystyle{ \sqrt{ \frac{1 + \cos x}{2} } }$;
$\tan \frac{x}{2} = \pm \displaystyle{ \sqrt{ \frac{1 - \cos x}{1 + \cos x} } }$.

% #41 ====================
\item
$\sin x + \sin y = 2 \sin \frac{1}{2} \left(x + y\right) \cos \frac{1}{2} \left(x - y\right)$.

% #42 ====================
\item
$\sin x - \sin y = 2 \cos \frac{1}{2} \left(x + y\right) \sin \frac{1}{2} \left(x - y\right)$.

% #43 ====================
\item
$\cos x + \cos y = -2 \cos \frac{1}{2} \left(x + y\right) \cos \frac{1}{2} \left(x - y\right)$.

% #44 ====================
\item
$\cos x - \cos y = -2 \sin \frac{1}{2} \left(x + y\right) \sin \frac{1}{2} \left(x - y\right)$.

% #45 ====================
\item
$\displaystyle{ \frac{a}{\sin A} = \frac{b}{\sin B} = \frac{c}{\sin C}}$; Law of Sines.

% #46 ====================
\item
$a^2 = b^2 + c^2 - 2bc \cos A$; Law of Cosines.

% #47 ====================
\item
$d = \sqrt{\left(x_1 - x_2\right)^2 + \left(y_1 - y_2\right)^2}$;
distance between points $\left(x_1, y_1\right)$ and $\left(x_2, y_2\right)$.

% #48 ====================
\item
$d = \displaystyle{ \frac{Ax_1 + By_1 + C}{\pm \sqrt{A^2 + B^2}}}$;
distance from line $Ax + By + C = 0$ to $\left(x_1, y_1\right)$.

% #49 ====================
\item
$x = \displaystyle{ \frac{x_1 + x_2}{2} }$, $y = \displaystyle{ \frac{y_1 + y_2}{2} }$; 
co\"ordinates of middle point.

% #50 ====================
\item
$x = x_0 + x'$, $y = y_0 + y'$; ransforming to new origin $\left(x_0, y_0\right)$.

% #51 ====================
\item
$x = x' \cos \theta - y' \sin \theta$, $y = x' \sin \theta + y' \cos \theta$;
transforming to new axes making the angle $\theta$ with old.

% #52 ====================
\item
$x = \rho \cos \theta$, $y = \rho \sin \theta$; transforming from rectangular to polar
co\"ordinates.

% #53 ====================
\item
$\rho = \sqrt{x^2 + y^2}$, $\theta = \arctan \frac{y}{x}$; transforming from polarto
rectangular co\"ordinates.

% #54 ====================
\item
Different forms of equation of a straight line:
  \begin{itemize}
  \item[(a)]
  $\displaystyle{ \frac{y - y_1}{x - x_1} = \frac{y_2 - y_1}{x_2 - x_1}}$,
  two-point form;

  \item[(b)]
  $\frac{x}{a} + \frac{y}{b} = 1$;
  intercept form;

  \item[(c)]
  $y - y_1 = m\left(x - x_1\right)$;
  slope-point form;

  \item[(d)]
  $y = mx + b$,
  slope-intercept form;

  \item[(e)]
  $x \cos \alpha + y \sin \alpha = p$,
  normal form;

  \item[(f)]
  $Ax + By + C = 0$,
  general form.

  \end{itemize}

% $55 ====================
\item
$\displaystyle{\tan \theta = \frac{m_1 - m_2}{1 + m_1 m_2}}$, angle between two
lines whose slopes are $m_1$ and $m_2$.
\[
\begin{aligned}
m_1 & = m_2 \textnormal{ when lines are parallel, and}\\
m_1 & = - \frac{1}{m_2} \textnormal{ when lines are perpendicular.}\\
\end{aligned}
\]

% #56 ====================
\item
$\left(x = \alpha\right)^2 + \left(y - \beta\right)^2 = r^2$, equation of circle with center $\left(\alpha,\beta\right)$ and radius $r$.

\end{enumerate}

Many of these facts are already know to \SAGE:

\begin{Verbatim}[fontsize=\scriptsize,fontfamily=courier,fontshape=tt,frame=single,label=\SAGE]

sage: a,b = var("a,b")
sage: log(sqrt(a))
log(a)/2
sage: log(a/b).simplify_log()
log(a) - log(b)
sage: sin(a+b).simplify_trig()
cos(a)*sin(b) + sin(a)*cos(b)
sage: cos(a+b).simplify_trig()
cos(a)*cos(b) - sin(a)*sin(b)
sage: (a+b)^5
(b + a)^5
sage: expand((a+b)^5)
b^5 + 5*a*b^4 + 10*a^2*b^3 + 10*a^3*b^2 + 5*a^4*b + a^5

\end{Verbatim}

\noindent
``Under the hood'' \SAGE used Maxima to do this simplification.

% ========================================
\section{Greek Alphabet}
\label{sec:1:2:greek-alphabet}

\begin{center}
\begin{tabular}{cc|cc|cc}
{\bf Letters}     & {\bf Names} & {\bf Letters}     & {\bf Names} & {\bf Letters}   & {\bf Names} \\ \hline
A$,\alpha$        & alpha       & I$,\iota$         & iota        & P$,\rho$        & rho         \\
B$,\beta$         & beta        & K$,\kappa$        & kappa       & $\Sigma,\sigma$ & sigma       \\
$\Gamma, \gamma$  & gamma       & $\Lambda,\lambda$ & lambda      & T$,\tau$        & tau         \\
$\Delta,\delta$   & delta       & M$,\mu$           & mu          & Y$,\upsilon$    & upsilon     \\
E$,\epsilon$      & epsilon     & N$, \nu$          & nu          & $\Phi,\phi$     & phi         \\
Z$,\zeta$         & zeta        & $\Xi,\xi$         & xi          & X$,\chi$        & chi         \\
H$,\eta$          & eta         & O$,o$             & omicron     & $\Psi,\psi$     & psi         \\
$\Theta,\theta$   & theta       & $\Pi,\pi$         & pi          & $\Omega,\omega$ & omega       \\
\end{tabular}
\end{center}

% ========================================
\section{Rules for Sings of the Trigonometric Functions}
\label{sec:1:3:rules-for-signs-of-the-trigonometric-functions}

\begin{center}
\begin{tabular}{l|llllll}
Quadrant & Sin & Cos & Tan & Cot & Sec & Csc \\ \hline
First    &	+  &	+  &	+  &	+  &	+  &	+  \\
Second   &  +  &	-  &	-  &	-  &	-  &	+  \\
Third    &	-  &	-  &	+  &	+  &	-  &	-  \\
Fourth   &  -  &	+  &	-  &	-  &	+  &  -  \\
\end{tabular}
\end{center}

% ========================================
\section{Natural Values of the Trigonometric Functions}
\label{sec:1:3:natural-values-of-the-trigonometric-functions}

\begin{center}
\begin{tabular}{cccccccc}
Angle in    &  Angle in &                &                &                &                &                  &                  \\
Radians     &   Degrees &  Sin           &  Cos           & Tan            & Cot            &   Sec            & Csc              \\  \hline
0           & 0         &  0             & 1              & 0              & $\infty$       & 1                & $\infty$         \\
$\pi / 6$   & 30        & $1 / 2$        & $\sqrt{3} / 2$ & $\sqrt{3} / 3$ & $\sqrt{3}$     & $2 \sqrt{3} / 3$ & 2                \\
$\pi / 4$   & 45        & $\sqrt{2} / 2$ & $\sqrt{2} / 2$ & 1              & 1              & $\sqrt{2}$       & $\sqrt{2}$       \\
$\pi / 3$   & 60        & $\sqrt{3} / 2$ & $1 / 2$        & $\sqrt{3}$     & $\sqrt{3} / 3$ & 2                & $2 \sqrt{3} / 3$ \\
$\pi / 2$   & 90        & 1              & 0              & $\infty$       & 0              & $\infty$         & 1                \\
$\pi$       & 180       & 0              & -1             & 0              & $\infty$       & -1               & $\infty$         \\
$3 \pi / 2$ & 270       & -1             & 0              & $\infty$       & 0              & $\infty$         & -1               \\
$2 \pi$     & 360       & 0              & 1              & 0              & $\infty$       & 1                & $\infty$         \\
\end{tabular}
\end{center}

\begin{center}
\begin{tabular}{ccllllcc}
Angle in & Angle in &        &        &        &        &          &          \\
Radians  & Degrees  &  Sin   &  Cos   & Tan    & Cot    &          &          \\  \hline
.0000    &	0       & .0000  & 1.0000 & .0000  & Inf.   & 90       & 1.5708   \\
.0175    &	1       & .0175  & .9998  & .0175  & 57.290 & 89       & 1.5533   \\
.0349    &	2       & .0349  & .9994  & .0349  & 28.636 & 88       & 1.5359   \\
.0524    &	3       & .0523  & .9986  & .0524  & 19.081 & 87       & 1.5184   \\
.0698    &	4       & .0698  & .9976  & .0699  & 14.300 & 86       & 1.5010   \\
.0873    &	5       & .0872  & .9962  & .0875  & 11.430 & 85       & 1.4835   \\
.1745    &	10      & .1736  & .9848  & .1763  & 5.671  & 80       & 1.3963   \\
.2618    &	15      & .2588  & .9659  & .2679  & 3.732  & 75       & 1.3090   \\
.3491    &	20      & .3420  & .9397  & .3640  & 2.747  & 70       & 1.2217   \\
.4863    &	25      & .4226  & .9063  & .4663  & 2.145  & 65       & 1.1345   \\
.5236    &	30      & .5000  & .8660  & .5774  & 1.732  & 60       & 1.0472   \\
.6109    &	35      & .5736  & .8192  & .7002  & 1.428  & 55       & .9599    \\
.6981    &	40      & .6428  & .7660  & .8391  & 1.192  & 50       &	.8727   \\
.7854    &	45      & .7071  & .7071  & 1.0000 & 1.000  & 45       & .7854    \\  \hline
         &          &        &        &        &        & Angle in & Angle in \\
         &          &  Cos   &  Sin   & Cot    & Tan    & Degrees  & Radians  \\  
\end{tabular}
\end{center}

You can create a table like this in \sage :

\begin{Verbatim}[fontsize=\scriptsize,fontfamily=courier,fontshape=tt,frame=single,label=\sage]

sage: RR15 = RealField(15)
sage: rads1 = [n*0.0175 for n in range(1,6)]
sage: rads2 = [0.0875+n*0.0875 for n in range(1,9)]
sage: rads = rads1+rads2
sage: trigs = ["sin", "cos", "tan", "cot"]
sage: tbl = [[eval(x+"(%s)"%y) for x in trigs] for y in rads]
sage: tbl = [[RR15(eval(x+"(%s)"%y)) for x in trigs] for y in rads]
sage: print Matrix(tbl)
[0.01750  0.9998 0.01750   57.14]
[0.03499  0.9994 0.03502   28.56]
[0.05247  0.9986 0.05255   19.03]
[0.06994  0.9976 0.07011   14.26]
[0.08739  0.9962 0.08772   11.40]
[ 0.1741  0.9847  0.1768   5.656]
[ 0.2595  0.9658  0.2687   3.722]
[ 0.3429  0.9394  0.3650   2.740]
[ 0.4237  0.9058  0.4677   2.138]
[ 0.5012  0.8653  0.5792   1.726]
[ 0.5749  0.8182  0.7026   1.423]
[ 0.6442  0.7648  0.8423   1.187]
[ 0.7086  0.7056   1.004  0.9958]

\end{Verbatim}

\noindent
The first column are the values of $\sin(x)$ at 
$x\in \{0.01750, 0.03500, ... %0.05250, 0.07000, 0.08750, 0.1750, 0.2625, 
%0.3500, 0.4375, 0.5250, 0.6125, 0.7000, 
0.7875\}$ (measured in radians). The second, third and fourth 
rows are the corresponding values for $\cos$, $\tan$ and $\cot$,
resp.

