 

\chapter{Collection of formulas}
\label{ch:1}

%\begin{enumerate}
%\item
%1. 
\section{Formulas for reference}
\label{sec:1}

For the convenience of the student we give 
the following list of elementary formulas from Algebra, Geometry, 
Trigonometry, and Analytic Geometry.

\begin{enumerate}
\item
% 1. 
Binomial Theorem (n being a positive integer):

\[
\begin{matrix} 
(a + b)^n = a^n + na^{n-1}b &+& \frac{n(n - 1)}{2!}a^{n-2}b^2 
+ \frac{n(n - 1)(n - 2)}{3!}a^{n-3}b^3 + \cdots \\ 
&+& \frac{n(n - 1)(n - 2)\cdots(n - r + 2)}{(r - 1)!}a^{n-r+1}b^{r-1} + \cdots 
\end{matrix}
\]

\item
% 2. 
$n! = 1 \cdot 2 \cdot 3 \cdot 4 \cdots (n - 1)n$.

\item
%3. 
In the quadratic equation $ax^2 + bx + c = 0$,

\qquad      when $b^2-4ac > 0$, the roots are real and unequal;

\qquad      when $b^2-4ac = 0$, the roots are real and equal;

\qquad      when $b^2-4ac < 0$, the roots are imaginary.

\item
% 4. 
When a quadratic equation is reduced to the form $x^2 + px + q = 0$,

\qquad     $p$ = sum of roots with sign changed, and 

\qquad  $q$ = product of roots.

\item
%  5. %% modified from the original %%
In an arithmetical series, $a$, $a+d$, $a+2d$, ..., 

\[
s = \sum_{i=0}^{n-1} a + id = \frac{n}{2}[2a + (n-1)d].
\]

\item
%  6.  %% modified from the original %%
In a geometrical series, $a$, $ar$, $ar^2$, ...,
 
\[
s = \sum_{i=0}^{n-1} ar^{i} = \frac{a(r^n - 1)}{r - 1}.
\]

\item
%  7. 
$\log ab = \log a + \log b$.

 \item
%  8. 
$\log \frac{a}{b} = \log a - \log b$.

\item
%  9. 
$\log a^n = n\log a$.

\item
% 10. 
$\log \sqrt[n]{a} = \frac{1}{n} \log a$.

\item
% 11. 
$\log 1 = 0$.

\item
% 12. 
$\log e = 1$.

\item
% 13. 
$\log \frac{1}{a} = -\log a$.

\item \footnote{In formulas 14-25, $r$ denotes radius, 
$a$ altitude, $B$ area of base, and $s$ slant height.}
%  14. 
Circumference of circle = $2 \pi\, r$. 

\item
% 15. 
Area of circle = $\pi\, r^2$.

\item
%  16. 
Volume of prism = $Ba$.
 
\item
% 17. 
Volume of pyramid = $\frac{1}{3} Ba$.

\item
%  18. 
Volume of right circular cylinder = $\pi\, r^2a$.
 
\item
% 19. 
Lateral surface of right circular cylinder = $2 \pi\, ra$.

\item
%  20. 
Total surface of right circular cylinder = $2 \pi\, r(r + a)$.

\item
% 21. 
Volume of right circular cone = $2 \pi\, r(r + a)$.

\item
%   22. 
Lateral surface of right circular cone = $\pi\ rs$.

\item
%   23. 
Total surface of right circular cone = $\pi\ r(r + s)$.

\item
%   24. 
Volume of sphere = $\frac{4}{3}\pi\ r^3$.

 \item
%  25. 
Surface of sphere = $4\pi\ r^2$.

\item
%   26. 
$\sin x = \frac{1}{\csc x}$; 

$\cos x = \frac{1}{\sec x}$; 

$\tan x = \frac{1}{\cot x}$.

\item
%   27. 
$\tan x = \frac{\sin{x}}{\cos{x}}$;
 
$\cot{x} = \frac{\cos{x}}{\sin{x}}$.

 \item
%  28. 
$\sin^2 x + \cos^2 x = 1$;

$1 + \tan^2 x = \sec^2 x$;

$1 + \cot^2 x = \csc^2 x$.

\item
%   29. 
$\sin x = \cos \left ( \frac{\pi}{2} - x \right )$; 

$\cos x = \sin \left ( \frac{\pi}{2} - x \right)$; 

$\tan x = \cot \left ( \frac{\pi}{2} - x \right )$.

\item
%   30. 
$\sin(\pi\ - x) = \sin x$; 

$\cos(\pi\ - x) = -\cos x$; 

$\tan(\pi\ - x) = -\tan x$.

 \item
%  31. 
$\sin (x + y) = \sin x \cos y + \cos x \sin y$.

\item
%   32. 
$\sin (x-y) = \sin x \cos y-\cos x \sin y$.

\item
%   33. 
$\cos(x \pm y) = \cos x \cos y +\mp \sin x \sin y$

 \item
%  34. 
$\tan(x + y) = \frac{\tan x + \tan y}{1 - \tan x \tan y}$.

\item
%   35. 
$\tan(x - y) = \frac{\tan x - \tan y}{1 + \tan x \tan y}$.

 \item
%  36. 
$\sin 2x = 2 \sin x \cos x$; 
$\cos 2x = \cos^2 x - \sin^2 x$; 
$\tan 2x = \frac{2 \tan x}{1 - \tan^2 x}$.

\item
%   37. 
$\sin x = 2\sin \frac{x}{2} \cos \frac{x}{2}$; 
$\cos x = \cos^2 \frac{x}{2} - \sin^2 \frac{x}{2}$; 
$\tan x = \frac{2 \tan \frac{1}{2} x}{1 - \tan^2 \frac{1}{2} x}$.

\item
%   38. 
$\cos^2 x = \frac{1}{2} + \frac{1}{2} \cos 2x$; 
$\sin^2 x = \frac{1}{2} - \frac{1}{2} \cos 2x$.

\item
%   39. 
$1 + \cos x = 2 \cos^2 \frac{x}{2}$; 
$1 - \cos x = 2 \sin^2 \frac{x}{2}$.

 \item
%  40. 
$\sin \frac{x}{2} = \pm \sqrt{ \frac{1 - \cos x}{2} }$; 
$\cos x/2 = \pm \sqrt{ \frac{1 + \cos x}{2} }$; 
$\tan \frac{x}{2} = \pm \sqrt{ \frac{1 - \cos x}{1 + \cos x}}$.

\item
%   41. 
$\sin x + \sin y = 2 \sin \frac{1}{2} (x + y) cos \frac{1}{2} (x - y)$.

\item
%   42. 
$\sin x - \sin y = 2 \cos \frac{1}{2} (x + y) sin \frac{1}{2} (x - y)$.

\item
%   43. 
$\cos x + \cos y = -2 \cos \frac{1}{2} (x + y) cos \frac{1}{2} (x - y)$.

\item
%   44. 
$\cos x - \cos y = -2 \sin \frac{1}{2} (x + y) sin \frac{1}{2} (x - y)$.

\item
%   45. 
$\frac{a}{\sin A} = \frac{b}{\sin B} = \frac{c}{\sin C}$; 
\mbox{ Law of Sines}.

 \item
%  46. 
$a^2 = b^2 + c^2 − 2bc\cos A$; \mbox{Law of Cosines}.

\item
%   47. 
$d = \sqrt{ (x_1 - x_2)^2 + (y_1 - y_2)^2}$; 
distance between points $(x_1,y_1)$ and $(x_2,y_2)$.

\item
%   48. 
$d = \frac{Ax_1 + By_1 + C}{\pm \sqrt{A^2 + B^2}}$; 
distance from line $Ax + By + C = 0$ to $(x_1,y_1)$.

 \item
%  49. 
$x = \frac{x_1 + x_2}{2}$, $y = \frac{y_1 + y_2}{2}$; 
coordinates of middle point.

\item
%   50. 
$x = x_0 + x'$, $y = y_0 + y'$; transforming to new origin $(x_0,y_0)$.

\item
%   51. 
$x = x' \cos \theta\ - y' \sin \theta\ $, 
$y = x' \sin \theta\ + y' \cos \theta$; 
transforming to new axes making the angle theta with old.

\item
%   52. 
$x = \rho\ \cos \theta\ $, $y = \rho\ \sin \theta$; 
transforming from rectangular to polar coordinates.

 \item
%  53. 
$\rho\ = \sqrt{x^2 + y^2}$, $\theta\ = \arctan \frac{y}{x}$; 
transforming from polar to rectangular coordinates.

\item
% 54. 
Different forms of equation of a straight line:

\begin{itemize}
\item[(a)] 
$\frac{y - y_1}{x - x_1} = \frac{y_2 - y_1}{x_2 - x_1}$, 
\mbox{ two-point form};

\item[(b)] 
$\frac{x}{a} + \frac{y}{b} = 1$, 
\mbox{ intercept form};

\item[(c)] 
$y-y_1 = m(x-x_1)$, 
\mbox{slope-point form};

\item[(d)] 
$y = mx + b$, 
\mbox{slope-intercept form};

\item[(e)] 
$x \cos \alpha\ + y \sin \alpha\ = p$, 
\mbox{ normal form};

\item[(f)] 
$Ax + By + C = 0$, general form.
\end{itemize}

\item
% 55. 
$\tan \theta\ = \frac{m_1 - m_2}{1 + m_1 m_2}$, 
angle between two lines whose slopes are $m_1$ and $m_2$.

\qquad    $m_1 = m_2$ when lines are parallel, and

\qquad     $m_1 = -\frac{1}{m_2}$ when lines are perpendicular.

\item
% 56. 
$(x - \alpha)^2 + (y - \beta)^2 = r^2$, equation of circle 
with center $(\alpha,\beta)$ and radius $r$.

\end{enumerate}


Many of these facts are already known to \sage :

\vskip .1in

\begin{Verbatim}[fontsize=\scriptsize,fontfamily=courier,fontshape=tt,frame=single,label=\sage]

sage: a,b = var("a,b")
sage: log(sqrt(a))
log(a)/2
sage: log(a/b).simplify_log()
log(a) - log(b)
sage: sin(a+b).simplify_trig()
cos(a)*sin(b) + sin(a)*cos(b)
sage: cos(a+b).simplify_trig()
cos(a)*cos(b) - sin(a)*sin(b)
sage: (a+b)^5
(b + a)^5
sage: expand((a+b)^5)
b^5 + 5*a*b^4 + 10*a^2*b^3 + 10*a^3*b^2 + 5*a^4*b + a^5

\end{Verbatim}

\vskip .1in
\noindent
``Under the hood'' \sage used Maxima to do this simplification.


%\item
% 2
\section{Greek alphabet}

\begin{center}
\begin{tabular}{cc|cc}
{\bf letters} & {\bf names} & {\bf letters} & {\bf names} \\ \hline
$A,\alpha$ & alpha & $N, \nu$ & nu \\
$B,\beta$ & beta & $\Xi,\xi$ & xi \\
$\Gamma, \gamma$ & gamma & $O,o$ & omicron \\
$\Delta,\delta$ & delta & $\Pi,\pi$ & pi \\
$E,\epsilon$ & epsilon & $P,\rho$ & rho \\
$Z,\zeta$ & zeta & $\Sigma,\sigma$ & sigma \\
$H,\eta$ & eta & $T,\tau$ & tau \\
$\Theta,\theta$ & theta & $Y,\upsilon$ & upsilon \\
$I,\iota$ & iota & $\Phi,\phi$ & phi \\
$K,\kappa$ & kappa & $X,\chi$ & chi \\
$\Lambda,\lambda$ & lambda & $\Psi,\psi$ & psi \\
$M,\mu$ & mu & $\Omega,\omega$ & omega \\
\end{tabular}
\end{center}


%\item
% 3
\section{Rules for signs of the trigonometric functions}

\begin{center}
\begin{tabular}{l|llllll}
Quadrant & Sin & Cos & Tan & Cot & Sec & Csc\\ \hline
First &	+ &	+ &	+ &	+ &	+ &	+\\
Second  & + &	-  &	-  &	-  &	-  &	+\\
Third &	-  &	-  &	+  &	+  &	-  &	-\\
Fourth  & -  &	+  &	-  &	-  &	+  & - \\
\end{tabular}
\end{center}

%\item
% 4
\section{Natural values of the trigonometric functions}
\label{sec:4}

\begin{center}
\begin{tabular}{ccllllll}
Angle in        &  Angle in &      &       &         &     &       &      \\
Radians         & Degrees   &  Sin  &  Cos & Tan     & Cot   &   Sec & Csc \\  \hline
0               &  0        &  0     &   1  &  0      & $\infty$  &   1   & $\infty$\\
$\frac{\pi}{6}$ & 30 & $\frac{1}{2}$ & $\frac{\sqrt{3}}{2}$ & $\frac{\sqrt{3}}{3}$ & $\sqrt{3}$ & $\frac{2\sqrt{3}}{3}$ & $2$ \\
$\frac{\pi}{4}$ & $45$ & $\frac{\sqrt{2}}{2}$ & $\frac{\sqrt{2}}{2}$ & 1 & 1 & $\sqrt{2}$ & $\sqrt{2}$\\
$\frac{\pi}{3}$ & 60 & $\frac{\sqrt{3}}{2}$ & $\frac{1}{2}$ & $\sqrt{3}$ & $\frac{\sqrt{3}}{3}$ & 2 & $\frac{2\sqrt{3}}{3}$ \\
$\frac{\pi}{2}$ &  90       &  1    &   0  & $\infty$ & 0        & $\infty$ & 1 \\
$\pi$           & 180       & 	0   &  -1  &   0      & $\infty$ & -1 & $\infty$\\
$\frac{3\pi}{2}$& 270       & -1   &    0  & $\infty$ &	0 & $\infty$  & -1\\
$2\pi$         &  360       &  0   &    1  &  0       & $\infty$ & 1  & $\infty$\\
\end{tabular}
\end{center}

{\small{
\begin{center}
\begin{tabular}{ccllllcc}
Angle in   &  Angle in &       &        &       &        &        &         \\
Radians    & Degrees   &  Sin  &  Cos   & Tan   & Cot    &        &         \\  \hline
.0000      &	0      &.0000  & 1.0000 & .0000 & Inf.   & 90     &  1.5708  \\
.0175      &	1      &.0175  & .9998  & .0175 & 57.290 & 89     &  1.5533  \\
.0349      &	2      &.0349  & .9994  & .0349 & 28.636 & 88     &  1.5359  \\
.0524      &	3      &.0523  & .9986  & .0524 & 19.081 & 87     &  1.5184  \\
.0698      &	4      &.0698  & .9976 & .0699 & 14.300 & 86      & 	1.5010  \\
.0873      &	5      &.0872  & .9962 & .0875 & 11.430 & 85      & 	1.4835  \\
.1745      &	10     &.1736  & .9848 & .1763 & 5.671  & 80      & 	1.3963  \\
.2618      &	15     &.2588  & .9659 & .2679 & 3.732  & 75      & 	1.3090  \\
.3491      &	20     &.3420  & .9397 & .3640 & 2.747  & 70      & 	1.2217  \\
.4863      &	25     &.4226  & .9063 & .4663 & 2.145  & 65      & 	1.1345  \\
.5236      &	30     &.5000  & .8660 & .5774 & 1.732  & 60      & 	1.0472  \\
.6109      &	35     &.5736  & .8192 & .7002 & 1.428  & 55      & 	.9599  \\
.6981      &	40     &.6428  & .7660 & .8391 & 1.192  & 50      &	.8727  \\
.7854      &	45     &.7071  & .7071 & 1.0000 & 1.000 & 45      & 	.7854 \\  \hline
           &           &       &       &        &       & Angle in  &  Angle in\\
           &           &  Cos  &  Sin  & Cot    & Tan   & Degrees   & Radians  \\  
\end{tabular}
\end{center}
}}

You can create a table like this in \sage :

\vskip .1in

\begin{Verbatim}[fontsize=\scriptsize,fontfamily=courier,fontshape=tt,frame=single,label=\sage]

sage: RR15 = RealField(15)
sage: rads1 = [n*0.0175 for n in range(1,6)]
sage: rads2 = [0.0875+n*0.0875 for n in range(1,9)]
sage: rads = rads1+rads2
sage: trigs = ["sin", "cos", "tan", "cot"]
sage: tbl = [[eval(x+"(%s)"%y) for x in trigs] for y in rads]
sage: tbl = [[RR15(eval(x+"(%s)"%y)) for x in trigs] for y in rads]
sage: print Matrix(tbl)
[0.01750  0.9998 0.01750   57.14]
[0.03499  0.9994 0.03502   28.56]
[0.05247  0.9986 0.05255   19.03]
[0.06994  0.9976 0.07011   14.26]
[0.08739  0.9962 0.08772   11.40]
[ 0.1741  0.9847  0.1768   5.656]
[ 0.2595  0.9658  0.2687   3.722]
[ 0.3429  0.9394  0.3650   2.740]
[ 0.4237  0.9058  0.4677   2.138]
[ 0.5012  0.8653  0.5792   1.726]
[ 0.5749  0.8182  0.7026   1.423]
[ 0.6442  0.7648  0.8423   1.187]
[ 0.7086  0.7056   1.004  0.9958]

\end{Verbatim}

\vskip .1in
\noindent
The first column are the values of $\sin(x)$ at 
$x\in \{0.01750, 0.03500, ... %0.05250, 0.07000, 0.08750, 0.1750, 0.2625, 
%0.3500, 0.4375, 0.5250, 0.6125, 0.7000, 
0.7875\}$ (measured in radians). The second, third and fourth 
rows are the corresponding values for $\cos$, $\tan$ and $\cot$,
resp..

\newpage
% 5
\section{Logarithms of numbers and trigonometric functions}

The common logarithm is the logarithm with base 10.
The fractional part of the logarithm of $x$, is known as the 
{\it mantissa} of the common logarithm of $x$.
For example, if $x=120$ then 
\[
    \log_{10}120=\log_{10}(10^2\times 1.2)=2+\log_{10}1.2\approx2+0.079181.
\]
so the very last number ($0.079181...$) is the mantissa. In the table below, this is
give simply as $0792$.

Table of mantissas of the common logarithms of numbers:
\vskip .2in

\begin{center}
\begin{tabular}{lllllllllll}
No. &  0    &  1     &  2     &  3     &  4     &  5     &  6     &  7     &  8     &  9 \\  \hline
1  &  0000  &  0414  &  0792  &  1139  &  1461  &  1761  &  2041  &  2304  &  2553  &  2788 \\
2  &  3010  &  3222  &  3424  &  3617  &  3802  &  3979  &  4150  &  4314  &  4472  &  4624 \\
3  &  4771  &  4914  &  5051  &  5185  &  5315  &  5441  &  5563  &  5682  &  5798  &  5911 \\
4  &  6021  &  6128  &  6232  &  6335  &  6435  &  6532  &  6628  &  6721  &  6812  &  6902 \\
5  &  6990  &  7076  &  7160  &  7243  &  7324  &  7404  &  7482  &  7559  &  7634  &  7709 \\
6  &  7782  &  7853  &  7924  &  7993  &  8062  &  8129  &  8195  &  8261  &  8325  &  8388 \\
7  &  8451  &  8513  &  8573  &  8633  &  8692  &  8751  &  8808  &  8865  &  8921  &  8976 \\
8  &  9031  &  9085  &  9138  &  9191  &  9243  &  9294  &  9345  &  9395  &  9445  &  9494 \\
9  &  9542  &  9590  &  9638  &  9685  &  9731  &  9777  &  9823  &  9868  &  9912  &  9956 \\
10  &  0000  &  0043  &  0086  &  0128  &  0170  &  0212  &  0253  &  0294  &  0334  &  0374 \\
11  &  0414  &  0453  &  0492  &  0531  &  0569  &  0607  &  0645  &  0682  &  07f9  &  0755 \\
12  &  0792  &  0828  &  0864  &  0899  &  0934  &  0969  &  1004  &  1038  &  1072  &  1106 \\
13  &  1139  &  1173  &  1206  &  1239  &  1271  &  1303  &  1335  &  1367  &  1399  &  1430 \\
14  &  1461  &  1492  &  1523  &  1553  &  1584  &  1614  &  1644  &  1673  &  1703  &  1732 \\
15  &  1761  &  1790  &  1818  &  1847  &  1875  &  1903  &  1931  &  1959  &  1987  &  2014 \\
16  &  2041  &  2068  &  2095  &  2122  &  2148  &  2175  &  2201  &  2227  &  2253  &  2279 \\
17  &  2304  &  2330  &  2355  &  2380  &  2405  &  2430  &  2455  &  2480  &  2504  &  2529 \\
18  &  2553  &  2577  &  2601  &  2625  &  2648  &  2672  &  2695  &  2718  &  2742  &  2765 \\
19  &  2788  &  2810  &  2833  &  2856  &  2878  &  2900  &  2923  &  2945  &  2967  &  2989 \\
\end{tabular}
\end{center}


\vskip .2in
The table of logarithms of the trigonometric functions 
(given in Granville's original text) is
omitted as they can be computed using a hand calculator or \sage.
