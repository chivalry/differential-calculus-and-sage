 
%last modified 2008-11-22

%\documentclass[12pt,runningheads]{book}
\documentclass[10pt,runningheads]{book}

\usepackage{fancyvrb}
\usepackage{moreverb}
\usepackage{amsfonts}
\usepackage{helvet}
\usepackage{fancyheadings}
\usepackage{amsbsy}
\usepackage{amsthm}
\usepackage{amsgen}
\usepackage{amsfonts}
%\usepackage[german,english]{babel}
\usepackage{array, rotating} 
\usepackage{graphicx, amsmath}
\usepackage{texdraw}
\usepackage{amssymb}
\usepackage{makeidx}  % allows for index generation
\usepackage[all,arc,dvips]{xy}
%\usepackage{twocolumn}
\usepackage{color}

\definecolor{DarkOlive}{rgb}{0.1047,0.2412,0.0064}
\definecolor{FireBrick}{rgb}{0.5812,0.0074,0.0083}
\definecolor{RoyalBlue}{rgb}{0.0236,0.0894,0.6179}
\definecolor{RoyalGreen}{rgb}{0.0236,0.6179,0.0894}
\definecolor{RoyalRed}{rgb}{0.6179,0.0236,0.0894}
\definecolor{LightBlue}{rgb}{0.8544,0.9511,1.0000}
\definecolor{Black}{rgb}{0.0,0.0,0.0}
\definecolor{FuncColor}{rgb}{1.0,0.0,0.0}
\usepackage{hyperref}
\usepackage{url}
\usepackage{xspace}


%\pagestyle{fancyplain}
%\lhead[\fancyplain{}{\rightmark}]{\fancyplain{}}
%\rhead[\fancyplain{}]{\fancyplain{}{\rightmark}}

\setlength{\parindent}{8pt}

\pagenumbering{roman}
\pagestyle{empty}
%\setcounter{page}{4}

\newtheorem{theorem}{Theorem}[section]
\newtheorem{corollary}{Corollary}[section]
\newtheorem{conjecture}{Conjecture}[section]
\newtheorem{lemma}{Lemma}[section]
\newtheorem{proposition}{Proposition}[section]
\newtheorem{definition}{Definition}[section]
\newtheorem{example}{Example}[section]
\newtheorem{axiom}{Axiom}[section]
\newtheorem{remark}{Remark}[section]
\newtheorem{exercise}{Exercise}[section]
\newtheorem{ponderable}{Ponderable}[section]

\newcommand{\aaa}{\mathbb{A}}
\newcommand{\nnn}{\mathbb{N}}
\newcommand{\fff}{\mathbb{F}}
\newcommand{\qqq}{\mathbb{Q}}
\newcommand{\rrr}{\mathbb{R}}
\newcommand{\ccc}{\mathbb{C}}
\newcommand{\zzz}{\mathbb{Z}}
\newcommand{\ppp}{\mathbb{P}}
\newcommand{\hhh}{{\cal H}}
\newcommand{\pf}{{\bf Proof: \ }}
%\newcommand{\qed}{$\Box$}
\newcommand{\stab}{{\rm stab}}
\newcommand{\beqn}{\begin{equation}}
\newcommand{\eeqn}{\end{equation}}
\newcommand{\arccot}{{\rm arccot}}
\newcommand{\arcsec}{{\rm arcsec}}
\newcommand{\arccsc}{{\rm arccsc}}
\newcommand{\vers}{{\rm vers}}
\newcommand{\arcvers}{{\rm arcvers}}
\def\fps2{{\rm ft./(sec.)}^2} % \newcommand won't work here !?

\newcommand{\SAGE}{{\sf Sage}\xspace}
\newcommand{\sage}{\SAGE}

\makeindex

\begin{document}

\pagestyle{fancyplain}
\lhead[\fancyplain{}{\rightmark}]{\fancyplain{}}
\rhead[\fancyplain{}]{\fancyplain{}{\rightmark}}

\author{William Anthony Granville, \\
with extra material added by David Joyner}
\title{Differential Calculus and \sage\\
{\footnotesize (revised edition)}}

\date{2009-1-10}

\maketitle


\vskip .1in

\addtocounter{page}{3}


\newpage
\addtocounter{page}{1}
\begin{center}
\large{
{\bf Contents}
}
\end{center}
\vskip .3in

\newpage
\tableofcontents

\newpage

\addtocounter{chapter}{-1}

\addtocounter{page}{1}
\input granville-calculus-preface.tex


\newpage

\pagenumbering{arabic}
\input granville-calculus-ch2.tex

\input granville-calculus-ch3.tex

\input granville-calculus-ch4.tex

\input granville-calculus-ch5.tex

\input granville-calculus-ch6.tex

\input app-newtons-method-ch6.tex

\input granville-calculus-ch7.tex

\input granville-calculus-ch8.tex

\input granville-calculus-ch9.tex

\input granville-calculus-ch10.tex

\input granville-calculus-ch11.tex

\input granville-calculus-ch12.tex

\input granville-calculus-ch13.tex

\input app-taylor-ch13.tex

\input granville-calculus-ch14.tex

\input granville-calculus-ch1.tex

\input granville-calculus-ref.tex

%\newpage

%\ \ 

%\newpage

\printindex
\end{document}

