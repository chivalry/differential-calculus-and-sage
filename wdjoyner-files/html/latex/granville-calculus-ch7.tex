
\chapter{Successive differentiation}

 
%74. 
\section{Definition of successive derivatives}

We have seen that the derivative of a function of $x$ 
is in general also a function of $x$. This new function may 
also be differentiable, in which case the derivative of 
the first derivative is called the second derivative of 
the original function. Similarly, the derivative of the 
second derivative is called the third derivative; and 
so on to the $n$-th derivative. Thus, if

\[
\begin{array}{rl}
\ y 	&=\ 3x^4,\\
\frac{dy}{dx} 	&=\ 12x^3,\\
\frac{d}{dx} \left ( \frac{dy}{dx} \right ) 	&=\ 36x^2,\\
\frac{d}{dx} \left [ \frac{d}{dx} \left ( \frac{dy}{dx} \right ) \right ] 	
&=\ 72x,
\end{array}
\]
and so on.

%75. 
\section{Notation}

The symbols for the successive derivatives are usually abbreviated as follows:

\[
\begin{array}{rl}
\frac{d}{dx} \left ( \frac{dy}{dx} \right ) 	&=\ \frac{d^2y}{dx^2},\\
\frac{d}{dx} \left [ \frac{d}{dx} \left ( \frac{dy}{dx} \right ) \right ] 	
&=\ \frac{d}{dx} \left ( \frac{d^2y}{dx^2} \right ) = \frac{d^3y}{dx^3},\\
. . . &	. . .\\
\frac{d}{dx} \left ( \frac{d^{n-1}y}{dx^{n-1}} \right )
 &	= \frac{d^ny}{dx^n}.
\end{array}
\]
If $y = f(x)$, the successive derivatives are also denoted by

\[
    f'(x),\ f''(x),\ f'''(x),\ f^{(4)}(x),\ ...,\ f^{(n)}(x);
\]
or
\[
    y',\ y'',\ y''',\ y^{(4)},\ ...,\ y^{(n)};
\]
or,
\[
\frac{d}{dx} f(x),\ \frac{d^2}{dx^2} f(x),
\ \frac{d^3}{dx^3} f(x),\ \frac{d^4}{dx^4} f(x),
\ ...,\ \frac{d^n}{dx^n} f(x). 
\]

%76. 

\section{The $n$-th derivative}

For certain functions a general expression involving $n$ 
may be found for the $n$-th derivative. The usual plan 
is to find a number of the first successive derivatives, as 
many as may be necessary to discover their law of formation, 
and then by induction write down the $n$-th derivative.

\begin{example}
{\rm
Given $y = e^{ax}$, find $\frac{d^n y}{dx^n}$.

Solution. $\frac{dy}{dx} = ae^{ax}$,
$\frac{d^2 y}{dx^2} 	= a^2e^{ax}$, \dots ,
$\frac{d^n y}{dx^n} = a^ne^{ax}$.

}
\end{example}

\begin{example}
{\rm
Given $y = \log\, x$, find $\frac{d^n y}{dx^n}$.

Solution. $\frac{dy}{dx} 	= \frac{1}{x}$,
$\frac{d^2 y}{dx^2} 	= -\frac{1}{x^2}$,
$\frac{d^3 y}{dx^3} 	= \frac{1 \cdot 2}{x^3}$,
$\frac{d^4 y}{dx^4} 	= \frac{1 \cdot 2 \cdot 3}{x^4}$,
\dots 
$\frac{d^n y}{dx^n} 	= (-1)^{n - 1} \frac{(n - 1)!}{x^n}$.
}
\end{example}

\begin{example}
{\rm
Given $y = \sin\, x$, find $\frac{d^n y}{dx^n}$.

Solution.
$\frac{dy}{dx} = \cos\, x = \sin \left ( x + \frac{\pi}{2} \right )$,

\[
\frac{d^2 y}{dx^2} 
= \frac{d}{dx} \sin \left ( x + \frac{\pi}{2} \right ) 	
= \cos \left ( x + \frac{\pi}{2} \right ) 
= \sin \left ( x + \frac{2 \pi}{2} \right ),
\]
\[
\frac{d^3 y}{dx^3} 
= \frac{d}{dx} \sin \left ( x + \frac{2 \pi}{2} \right ) 	
= \cos \left ( x + \frac{2 \pi}{2} \right ) 
= \sin \left ( x + \frac{3 \pi}{2} \right )
\]
\[
\dots
\]
\[
\frac{d^n y}{dx^n} 	= \sin \left ( x + \frac{n \pi}{2} \right ).
\]
}
\end{example}

%77. 
\section{Leibnitz's Formula for the $n$-th derivative of a product}

This formula expresses the $n$-th derivative of the 
product of two variables in terms of the variables 
themselves and their successive derivatives.

If u and v are functions of $x$, we have, from 
equation (V) in \S \ref{sec:33}  above, %V [§ 33],

\[
    \frac{d}{dx} (uv) = \frac{du}{dx} v + u \frac{dv}{dx}.
\]
Differentiating again with respect to $x$,

\[
\frac{d^2}{dx^2}(uv) 
= \frac{d^2u}{dx^2}v + \frac{du}{dx} \frac{dv}{dx} 
+ \frac{du}{dx} \frac{dv}{dx} + u \frac{d^2 v}{dx^2} 
= \frac{d^2 u}{dx^2}v + 2 \frac{du}{dx} \frac{dv}{dx} 
+ u \frac{d^2 v}{dx^2}.
\]
Similarly,

\[
\begin{array}{ll}
\frac{d^3}{dx^3}(uv) 	
&= \frac{d^3 u}{dx^3} + \frac{d^2 u}{dx^2} \frac{dv}{dx} 
+ 2\frac{d^2 u}{dx^2} \frac{dv}{dx} + 2 \frac{du}{dx} \frac{d^2 v}{dx^2} 
+ \frac{du}{dx} \frac{d^2 v}{dx^2} + u \frac{d^3 v}{dx^3}\\
&= \frac{d^3 u}{dx^3}v + 3\frac{d^2 u}{dx^2} \frac{dv}{dx} 
+ 3 \frac{du}{dx} \frac{d^2 v}{dx^2} + u \frac{d^3 v}{dx^3}.
\end{array}
\]
However far this process may be continued, it will be seen 
that the numerical coefficients follow the same law as 
those of the Binomial Theorem, and the indices of 
the derivatives correspond\footnote{To make this 
correspondence complete, $u$ and $v$ are considered as 
$\frac{d^0 u}{dx^0}$ and $\frac{d^0 v}{dx^0}$.} 
to the exponents of the Binomial Theorem.
Reasoning then by mathematical induction from the 
$m$-th to the $(m + 1 )$-st derivative of the product, 
we can prove {\it Leibnitz's Formula}
\index{Leibnitz's Formula}

\begin{equation}
%(17) 
\frac{d^n}{dx^n}(uv) 
= \frac{d^n u}{dx^n} v + n \frac{d^{n - 1} u}{dx^{n - 1}} \frac{dv}{dx} 
+ \frac{n (n - 1)}{2!} \frac{d^{n - 2} u}{dx^{n - 2}} \frac{d^2 v}{dx^2} 
+ \cdots + n \frac{du}{dx} \frac{d^{n - 1} v}{dx^{n - 1}} 
+ u \frac{d^nv}{dx^n},
\label{eqn:17-77}
\end{equation}

\begin{example}
{\rm
Given $y = e^x\log\, x$, find $\frac{d^3 y}{dx^3}$ by Leibnitz's Formula.

Solution. Let 	$u = e^x$, and 	$v = \log\, x$;
then 	
$\frac{du}{dx} = e^x$, 	$\frac{dv}{dx} = \frac{1}{x}$,
$\frac{d^2 u}{dx^2} = e^x$, $\frac{d^2 v}{dx^2} = - \frac{1}{x^2}$,
$\frac{d^3 u}{dx^3} = e^x$, $\frac{d^3 v}{dx^3} = \frac{2}{x^3}$.

Substituting in (\ref{eqn:17-77}), we get

\[
\frac{d^3 y}{dx^3} = e^x \log x + \frac{3 e^x}{x} - \frac{3 e^x}{x^2} = e^x \left ( \log x + \frac{3}{x} - \frac{3}{x^2} + \frac{2}{x^3} \right ).
\]

This can be verified using the \sage commands:


\vskip .1in

\begin{Verbatim}[fontsize=\small,fontfamily=courier,fontshape=tt,frame=single,label=\sage]

sage: x = var("x")
sage: f = exp(x)*log(x)
sage: diff(f,x,3)
e^x*log(x) + 3*e^x/x - 3*e^x/x^2 + 2*e^x/x^3

\end{Verbatim}
\vskip .1in

\noindent

}
\end{example}

\begin{example}
{\rm
Given $y = x^2e^{ax}$, find $\frac{d^n y}{dx^n}$ by Leibnitz's Formula.

Solution. Let $u = x^2$, and $v = e^{ax}$;
then $\frac{du}{dx} = 2x$, $\frac{dv}{dx} = ae^{ax}$,
$\frac{d^2 u}{dx^2} = 2x$, $\frac{d^2 v}{dx^2} = a^2 e^{ax}$,
$\frac{d^3 u}{dx^3} = 0$, $\frac{d^3 v}{dx^3} = a^3 e^{ax}$,
\dots,
$\frac{d^n u}{dx^n} = 0$, 
$\frac{d^n v}{dx^n} = a^n e^{ax}$.
Substituting in (\ref{eqn:17-77}), we get

\[
\frac{d^n y}{dx^n} 
= x^2 a^n e^{ax} + 2na^{n - 1} x e^{ax} + n(n - 1)a^{n - 2} e^{ax} 
= a^{n - 2} e^{ax} [ x^2 a^2 + 2nax + n(n - 1) ].
\]

}
\end{example}

%78. 
\section{Successive differentiation of implicit functions}

To illustrate the process we shall find 
$\frac{d^2 y}{dx^2}$ from the equation of the hyperbola

\[
    b^2x^2 - a^2y^2 = a^2b^2.
\]
Differentiating with respect to $x$, as in \S \ref{sec:63}, %§ 63, p. 69,

\[
    2b^2 x - 2a^2 y \frac{dy}{dx} = 0,
\]
or,

\begin{equation}
%(A) 
\frac{dy}{dx} = \frac{b^2 x}{a^2 y}.
\label{eqn:A-78}
\end{equation}
Differentiating again, remembering that $y$ is a function of $x$,

\[
    \frac{d^2 y}{dx^2} 
= \frac{a^2 y b^2 - b^2 x a^2 \frac{dy}{dx}}{a^4 y^2}.
\]
Substituting for $\frac{dy}{dx}$ its value from (\ref{eqn:A-78}),

\[
    \frac{d^2 y}{dx^2} 
= \frac{a^2 b^2 y - a^2 b^2 x \left ( \frac{b^2 y}{a^2 y} \right ) }{a^4 y^2} 
= -\frac{b^2 (b^2 x^2 - a^2 y^2)}{a^4 y^3}.
\]
The given equation, $b^2x^2 - a^2y^2 = a^2b^2$,
therefore gives, 

\[
\frac{d^2 y}{dx^2} = - \frac{b^4}{a^2 y^3}. 
\]

\sage can be made to do a lot of this work for you (though
the notation doesn't get any prettier):


\vskip .1in

\begin{Verbatim}[fontsize=\small,fontfamily=courier,fontshape=tt,frame=single,label=\sage]

sage: x = var("x")
sage: y = function("y",x)
sage: a = var("a")
sage: b = var("b")
sage: F = x^2/a^2 - y^2/b^2 - 1
sage: F.diff(x)
2*x/a^2 - 2*y(x)*diff(y(x), x, 1)/b^2
sage: F.diff(x,2)
-2*y(x)*diff(y(x), x, 2)/b^2 - 2*diff(y(x), x, 1)^2/b^2 + 2/a^2
sage: solve(F.diff(x) == 0, diff(y(x), x, 1))
[diff(y(x), x, 1) == b^2*x/(a^2*y(x))]
sage: solve(F.diff(x,2) == 0, diff(y(x), x, 2))
[diff(y(x), x, 2) == (b^2 - a^2*diff(y(x), x, 1)^2)/(a^2*y(x))]

\end{Verbatim}
\vskip .1in

\noindent
This basically says

\[
y'=\frac{d y}{dx} =  \frac{b^2x}{a^2 y},
\]
and
\[
y''=\frac{d^2 y}{dx^2} = - \frac{b^2-a^2(y')^2}{a^2 y}. 
\]
Now simply plug the first equation into the second,
obtaining $y''=-b^2\frac{1-a^{-2}b^2x^2/y^2}{a^2y}$.
Next, use the given equation in the form
$a^{-2}b^2x^2/y^2-1=b^2/y^2$ to get the result above.

\section{Exercises}

%Differentiate the following:
Verify the following derivatives:

\begin{enumerate}
\item
%1
$y = 4x^3 - 6x^2 + 4x + 7$.  	

Ans.	
$\frac{d^2 y}{dx^2} = 12(2x - 1)$.

\item
%2
$f(x) = \frac{x^3}{1 - x}$.  	

Ans.	
$f^{(4)}(x) = \frac{4!}{(1 - x)^5}$.

\item
%3
$f(y) = y^6$.  	

Ans.	$f^{(6)}(y) = 6!$.

\item
%4
$y = x^3\log\, x$.  	

Ans.	
$\frac{d^4 y}{dx^4} = \frac{6}{x}$.

\item
%5
$y = \frac{c}{x^n}$. 	
$y'' = \frac{n(n + 1)c}{x^{n + 2}}$.

\item
%6
$y = (x - 3)e^{2x} + 4xe^x + x$.  	

Ans.	
$y'' = 4e^x[(x - 2)e^x + x + 2]$.

\item
%7
$y = \frac{a}{2} (e^{\frac{x}{a}} + e^{-\frac{x}{a}})$.  	

Ans.	
$y'' = \frac{1}{2a} (e^{\frac{x}{a}} + e^{-\frac{x}{a}}) = \frac{y}{a^2}$.

\item
%8
$f(x) = ax^2 + bx + c$.  	

Ans.	
$f'''(x) = 0$.

\item
%9
$f(x) = \log(x + 1)$. 	 	

Ans.
$f^{(4)} (x) = -\frac{6}{(x + 1)^4}$.

\item
%10
$f(x) = log(e^x + e^{-x})$.  	

Ans.
$f'''(x) = -\frac{8(e^x - e^{-x})}{(e^x - e^{-x})^3}$.

\item
%11
$r = \sin\, a\theta$. 	 	

Ans.
$\frac{d^4 r}{d\theta^4} = a^4 \sin a\theta = a^4 r$.

\item
%12
$r = \tan\, \phi$.  	

Ans.	
$\frac{d^3 r}{d\phi^3} = 6 \sec^6 \phi - 4 \sec^2 \phi$.

\item
%13
$r = \log\sin\, \phi$.  	

Ans.	
$r''' = 2\cot\,\phi \csc^2\phi$.

\item
%14
$f(t) = e^{-t}\cos\, t$.  	

Ans.
$f^{(4)}(t) = - 4e^{- t}\cos\, t = - 4f(t)$.

\item
%15
$f(\theta) = \sqrt{\sec 2\theta}$.  	

Ans.	
$f''(\theta) = 3[f(\theta)]5 - f(\theta)$.

\item
%16
$p = (q^2 + a^2) \arctan \frac{q}{a}$.  	

Ans.	
$\frac{d^3 p}{dq^3} = \frac{4a^3}{(a^2 + q^2)^2}$.

\item
%17
$y = a^x$.  	

Ans.	
$\frac{d^n y}{dx^n} = (\log a)^n a^x$.

\item
%18
$y = \log(1 + x)$.  	

Ans.	
$\frac{d^n y}{dx^n} = (-1)^{n - 1} \frac{(n - 1)!}{(1 + x)^n}$.

\item
%19
$y = \cos\, ax$. 	

Ans.
$\frac{d^n y}{dx^n} = a^n \cos \left ( ax + \frac{n\pi}{2} \right )$.

\item
%20
$y = x^{n- 1}\log\, x$.  	

Ans.
$\frac{d^n y}{dx^n} = \frac{(n - 1)!}{x}$.

\item
%21
$y = \frac{1 - x}{1 + x}$.  	

Ans.	
$\frac{d^n y}{dx^n} = 2 (-1)^n \frac{n!}{(1 + x)^{n + 1}}$.

Hint: Reduce fraction to form $-1 + \frac{2}{1 + x}$ before differentiating.

\item
%22
If $y = e^x\sin\, x$, prove that 
$\frac{d^2 y}{dx^2} - 2 \frac{dy}{dx} + 2y = 0$.

\item
%23
If $y = a\cos(\log\, x) + b\sin(\log\, x)$, prove that 
$x^2 \frac{d^2 y}{dx^2} + x \frac{dy}{dx} + y = 0$.
\end{enumerate}

Use Leibnitz's Formula in the next four examples:

\begin{enumerate}
\addtocounter{enumi}{23}
\item
%24
$y = x^2a^x$. 	 	

Ans.
$\frac{d^n y}{dx^n} = a^x (\log a)^{n - 2} [(x \log a + n)^2 - n]$.

\item
%25
$y = xe^x$. 	 	

Ans.
$\frac{d^n y}{dx^n} = (x + n)e^x$.

\item
%26
$f(x) = e^x\sin\, x$. 	 	

Ans.
$f^{(n)}(x) = (\sqrt{2})^n e^x \sin \left ( x + \frac{n\pi}{4} \right )$.

\item
%27
$f(\theta) = \cos\,a\theta\cos\,b\theta$.  	

Ans.	
$f^{n}(\theta) 
= \frac{(a + b)^n}{2} \cos \left [ (a + b)\theta + \frac{n\pi}{2} \right ]
+ \frac{(a - b)^n}{2} \cos \left [ (a - b)\theta + \frac{n\pi}{2} \right ]$.

\item
%28
Show that the formulas for acceleration, 
(\ref{eqn:14-72}), (\ref{eqn:15-73}), %p. 92 [§71], 
may be written
$a = \frac{d^2 s}{dt^2}$, 
$a_x = \frac{d^2 x}{dt^2}$, 
$a_y = \frac{d^2 y}{dt^2}$.

\item
%29
$y^2 = 4ax$.  	

Ans.
$\frac{d^2 y}{dx^2} = -\frac{4a^2}{y^3}$.

\item
%30
$b^2x^2 + a^2y^2 = a^2b^2$.  	

Ans.	
$\frac{d^2 y}{dx^2} = -\frac{b^4}{a^2 y^3}$; 
$\frac{d^3 y}{dx^2} = -\frac{3b^6 x}{a^4 y^5}$.

\item
%31
$x^2 + y^2 = r^2$. 	
$\frac{d^2 y}{dx^2} = -\frac{r^2}{y^3}$.

\item
%32
$y^2 + y = x^2$. 	

Ans.
$\frac{d^3 y}{dx^3} = -\frac{24x}{(1 + 2y)^5}$.

\item
%33
$ax^2 + 2hxy + by^2 = 1$.  	

Ans.	
$\frac{d^2 y}{dx^2} = \frac{h^2 - ab}{(hx + by)^3}$.

\item
%34
$y^2 - 2xy = a^2$.  	

Ans.
$\frac{d^2 y}{dx^2} 
= \frac{a^2}{(y - x)^3}; \frac{d^3 y}{dx^3} = -\frac{3a^2 x}{(y - x)^5}$.

\item
%35
$\sec\,\phi\cos\,\theta = c$. 	 	

Ans.
$\frac{d^2 \theta}{d\phi^2} = \frac{\tan^2 \theta - \tan^2 \phi}{\tan^3 \theta}$.

\item
%36
$\theta=\tan(\phi + \theta)$. 	

Ans.
$\frac{d^3 \theta}{d\phi^3} = -\frac{2(5 + 8\theta^2 + 3\theta^4)}{\theta^8}$.

\item
%37
Find the second derivative in the following:

\[
\begin{array}{ll}
(a)\ \  \log(u + v) = u - v. &	(e)\ \  y^3 + x^3 - 3axy = 0.\\
(b)\ \  e^u + u = e^v + v.  &	(f)\ \  y^2 - 2mxy + x^2 - a = 0.\\
(c)\ \  s = 1 + te^s.  &	(g)\ \  y = \sin(x + y).\\
(d)\ \  e^s + st - e = 0.  &	(h)\ \  e^{x + y} = xy.\\
\end{array}
\]

\end{enumerate}

